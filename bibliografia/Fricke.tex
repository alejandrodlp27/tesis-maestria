\documentclass{article}
\usepackage{amsmath,amssymb,amsfonts}

\begin{document}

\section*{6. Zusammengesetzte ungerade Transformationsgrade.}

1. Transformation $15^{ten}$ Grades. Die zur linken Seite der imaginären w-Achse gelegene Hälfte des Klassenpolygons $K_15$ ist das in Fig. 22 dargestellte, von sechs Symmetriekreisen begrenzte Kreisbogensechseck. Die  beiden Ecken $e_0$ und $e_1$, bei 

$$w=...$$

und 

$$w=...$$

gelegen, sind die Fixpunkte der Substitutionen $matrix$ und $matrix$ ; sie sind die Nullpunkte der beiden qua dratischen Formen (15, 15, 4), (30, 15, 2), durch die wir die beiden Klassen der Diskriminante $D=-15$ repräsentieren können. Der Fixpunkt $e_2$ von $W_15$ gehört zur Hauptklasse der Diskriminante $D=-60$; und der $w=...$ gelegene Fixpunkt der Substitution $matrix$ ist der Nullpunkt der quadratischen Form (120, 90, 17), die die zweite Formklasse mit $D=-60$ repräsentiert. Die drei mit den Nummern 1, 2, 3 versehenen Seiten entsprechen den Gleichungen und Spiegelungen: 

$$1... $$
$$2... $$
$$3... $$

während der Kreis 4 der Symmetriekreis der Spiegelung $W_15$ ist. Die zweite und dritte Spiegelung sind bereits in der $\Gamma_15$ enthalten. Neben der Spitze $\infty$ ragt $K_15$ noch mit dem Spitzenzyklus $\pm1/3$ an die reelle w-Acbse heran. 

Funktionentheoretisch sind die zusammengesetzten Transformationsgrade besonders leicht zugänglich. Man setze zur Abkürzung: 

$$\Delta_\nu=...$$

und beachte, daß sowohl $\Delta_3.\Delta_5$ wie $\Delta_1.\Delta_15$ gegenüber den Substitutionen der $\Gamma_15$ invariant sind. Der Quotient $\Delta_3.\Delta_5/\Delta_3.\Delta_5$ ist also eine Funktion des Polygons $K_15$, und zwar kann diese Funktion Pole und Nullpunkte nur in der Spitze $i\infty$ und im Spitzenzyklus $\pm1/3$ haben. Da aber in der Spitze $i\infty$ ein Pol achter Ordnung liegt, so findet sich im Zyklus $\pm1/3$ ein Nullpunkt der gleichen Ordnung, so daß in: 

$$(1) \tau=...$$

bereits eine einwertige Funktion von $K_15$ gewonnen ist. 

Die beiden Formen $\sqrt[8]{\Delta_3\Delta_5}$ und $\sqrt[8]{\Delta_1\Delta_{15}}$ als solche der Dimension —3 haben beide Nullpunkte der Gesamtordnung 3 auf $K_15$. Dabei hat $\sqrt[8]{\Delta_3\Delta_5}$ in der Spitze $i\infty$ einen Nullpunkt erster Ordnung und also im Zyklus $\pm1/3$ einen solchen zweiter Ordnung, während $\sqrt[8]{\Delta_1\Delta_{15}}$ an diesen beiden Stellen bzw. Nullpunkte der Ordnung 2 und 1 hat. Hiernach haben wir in: 

$$z_0,z_1=...$$ 

zwei ganze Formen mit je einem Nullpunkte erster Ordnung im Zyklus $\pm1/3$ bzw. in der Spitze $i\infty$, deren Quotient $z_0:z_1$ die in (1) erklärte Funktion $\tau$ ist. In $(az_0+bz_1)$ aber haben wir eine Formenschar mit einem beweglichen Nullpunkte erster Ordnung auf $K_15$. 

Das Transformationspolygon $T_15$ wird durch $\tau(\omega)$ auf eine zweiblättrige Riemannsche Fläche des Geschlechtes 1 abgebildet, deren Yerzweigungsform wir in üblicher Art herstellen. Die nähere formentheoretische Diskussion zeigt, daß die zur $\Gamma_{\psi(15)}$ gehörende ganze Modulform 
$(-2)$ Dimension: 

$$(3) v=...$$

die gegenüber $W_{15}$ Zeichenwechsel erfährt, ihre vier einfachen Nullstellen in den Punkten hat, die die Verzweigungspunkte jener zweiblättrigen Fläche liefern. Die Potenzreihen ergeben dann für $v^2$ die Darstellung: 

$$(4) . v^2=... $$

womit die Verzweigungsform gewonnen ist. Als Funktionssystem des Transformationspolygons $T_{15}$ haben wir daraufhin:

$$(5) \tau=...,\sigma=...$$ 

wobei sich $\sigma$ in $\tau$ durch die Quadratwurzel darstellt: 

$$(6) \sigma=\sqrt{\tau^4+...}$$

so daß wir hier mit einem elliptischen Gebilde der absoluten Invariante $\frac{13^3 37^3}{2^6 3^7 5^4}$ tun haben.






Zur Darstellung von $J$ als rationale Funktion von $\sigma$ und $\tau$ bedienen wir uns der in (6) S.392 bei der Transformation fünften Grades eingeführten Funktion $\tau$, die bier mit $\tau_5$ bezeichnet werden möge, und in der sich $J$ in der Gestalt (1.3) S.393 darstellt. Es ist hinreichend, die auf $T_{15}$ vierwertige Funktion $\tau_5$ in $\sigma$ und $\tau$ darzustellen. Wird $\tau_5$ durch $W_{15}$ in $\tau'_5$ transformiert, so hat man: 

\begin{equation}
  \tau_5=125\sqrt[4]{\frac{\Delta_5}{\Delta_1}},\quad
  \tau'_5=125\sqrt[4]{\frac{\Delta_3}{\Delta_{15}}},\quad
\tau'_5-\tau_5=\frac{\sqrt[4]{\Delta_1\Delta_3}-125\sqrt[4]{\Delta_5\Delta_{15}}}{\sqrt[4]{\Delta_1\Delta_{15}}}
\end{equation}

Der rechts stehende Zähler ist nun eine ganze Form $(-6)^{ter}$ Dimension von $K_{15}$, die gegenüber $W_{15}$ Zeichenwechsel erfährt und also den Faktor $v$ enthält. Hierdurch werden Nullpunkte der Gesamtordung 2 auf $K_{15}$ erledigt, so daß noch solche der Gesamtordnung 4 übrigbleiben. Ein Nullpunkt erster Ordnung liegt in der Spitze $i\infty$, so daß die Ordnung 3 verbleibt. Da diese Ordnung ganzzahlig ist, so muß im Zyklus $\pm1/3$, wo unser Ausdruck sicher verschwindet, mindestens ein Nullpunkt erster Ordnung liegen. Zwei weitere Nullpunkte dieser Ordnung sind dann noch zu bestimmen. Dieser Überlegung entspricht der Ansatz: 
\[
  \sqrt[4]{\Delta_1\Delta_3}-125\sqrt[4]{\Delta_5\Delta_{15}}=v z_0 z_1(az_0^2+bz_0z_1+cz_1^2)
\]

Die Potenzreihen ergeben:

\begin{equation}
  \sqrt[4]{\Delta_1\Delta_3}-125\sqrt[4]{\Delta_5\Delta_{15}}=v z_0 z_1(z_0^2-4z_0z_1-z_1^2)
\end{equation}

Zur Prüfung dieses Ergebnisses berechne man mit Hilfe von (4) den Ausdruck von $(\sqrt[4]{\Delta_1\Delta_3}+125\sqrt[4]{\Delta_5\Delta_{15}})^2$ in $z_0$, $z_1$, wobei sich das Quadrat einer homogenen Funktion sechsten Grades ergeben muß. Dies bestätigt sich, man findet durch Ausziehen der Quadratwurzel: 

\begin{equation}
  \sqrt[4]{\Delta_1\Delta_3}+125\sqrt[4]{\Delta_5\Delta_{15}}=z_0 z_1(z_0^4-9z_0^3z_1-9z_0z_1^3-z_1^4).
\end{equation}

Von (7) und (8) aus gelangt man nun leicht zum Ziele: Der gesuchte Ausdruck der heim Grade 5 auftretenden Funktion $\tau_5$ in den jetzigen $\sigma$ und $\tau$ ist: 

\begin{equation}
  \tau_5=\frac{\tau^4-9\tau^3-9\tau-1-\sigma(\tau^2-4\tau-1)}{2\tau}
\end{equation}

\section*{6. Composite odd transformation line.}

Transformation $15^{th}$ degrees. The half of the class polygon $K_{15}$, which is to the left of the imaginary $\omega$-axis, is the circular hexagon delimited by six symmetry circles shown in FIG22. The two corners $e_0$ and $ e_1 $, at
$$\omega=\frac{-\sqrt{15}+i}{2\sqrt{15}}$$
and
$$\omega=\frac{-\sqrt{15}+i}{4\sqrt{15}}$$
located, the fixed points of the substitutions are $\begin{pmatrix}-15&-8\\30&15\end{pmatrix}$ and $\begin{pmatrix}-15&-4\\60&15\end{pmatrix}$; they are the zero points of the two quadratic forms (15, 15, 4), (30, 15, 2), by which we can represent the two classes of the discriminant $D=-15$. The fixed point $e_2$ of $W_{15}$ belongs to the main class of the discriminant $ D=-60$; and the $\omega=\frac{-3\sqrt{15}+i}{8\sqrt{15}}$ fixed point of the substitution $\begin{pmatrix}-45&-17\\120&45\end{pmatrix}$ is the zero point of the quadratic form (120, 90, 17), which represents the second shape class with $D=-60$. The three pages numbered 1, 2, 3 correspond to the equations and reflections:

$$1.\quad 30(\xi^2+\eta^2)+30\xi+7=0,\qquad \omega'=\frac{-15\bar{\omega}-7}{30\bar{\omega}+15}$$
$$2.\quad 15(\xi^2+\eta^2)+11\xi+2=0,\qquad \omega'=\frac{-11\bar{\omega}-4}{30\bar{\omega}+11}$$
$$3.\quad 15(\xi^2+\eta^2)+8\xi+1=0,\qquad \omega'=\frac{-4\bar{\omega}-1}{15\bar{\omega}+4}$$

while circle 4 is the symmetry circle of the reflection $\overline{W}_{15}$. The second and third reflections are already included in the $\Gamma_{\psi(15)}$. In addition to the peak $i\infty $, $K_{15}$ still reaches the real $\omega$-axis with the peak cycle $\pm1/3$. 

Functionally, the composite degrees of transformation are particularly easy to access. For short:

$$\Delta(v\omega_1,\omega_2)=\Delta_{\nu}$$

and note that both $ \Delta_3\cdot\Delta_5$ and $\Delta_1\cdot\Delta_{15}$ are invariant to the substitutions of $\Gamma_{15}$. The quotient $\Delta_3\cdot\Delta_5/\Delta_1\cdot\Delta_{15} $ is thus a function of the polygon $K_{15}$, and this function can only use poles and zeros in the peak $i\infty$ and in the peak cycle $\pm1/3$ have. But since in the peak $i\infty$ lies a pole of the eighth order, there is a zero point of the same order in the cycle $\pm1/3$, so that in:

\begin{equation}\label{1}
	\tau(\omega)=\sqrt[8]{\frac{\Delta_3\cdot\Delta_5}{\Delta_1\cdot\Delta_{15}}}=q^{-2}+3+9q^2+\cdots
\end{equation}

already won a one-valued function of $K_{15}$. 

The two forms $\sqrt[8]{\Delta_3\Delta_5}$ and $\sqrt[8]{\Delta_1\Delta_{15}}$ as those of dimension -3 both have zero points of order 3 on $K_{15}$. $\sqrt[8]{\Delta_3\Delta_5} $ has a first-order zero in the top $i\infty$ and thus such a second order in the $\pm1/3$ cycle, while $\sqrt[8]{\Delta_1\Delta_{15}}$ at these two places or zero points of order 2 and 1. After that we have in:

\begin{equation}\label{2}
	\begin{cases}
		z_0=\sqrt[24]{\frac{\Delta^2_3\Delta^2_5}{\Delta_1\Delta_{15}}}=\frac{2\pi}{\omega_2}(1+q^2+2q^4+q^6+3q^8+q^{10}+\cdots)\\
		z_1=\sqrt[24]{\frac{\Delta^2_1\Delta^2_{15}}{\Delta_3\Delta_{5}}}=\frac{2\pi}{\omega_2}(q^2-2q^4-q^6+3q^8-q^{10}+\cdots)
	\end{cases}
\end{equation}

two whole forms, each with a first-order zero in the cycle $\pm1/3$ or in the peak $i\infty $, whose quotient $z_0:z_1$ is the function $\tau $ declared in (1). But in $(az_0+bz_1)$ we have a set of shapes with a first-order moving zero on $K_{15}$. 

The transformation polygon $T_{15}$ is mapped by $\tau(\omega)$ to a two-leaf Riemann surface of gender 1 whose branching form we produce in the usual way. The closer theory-theoretical discussion shows that the whole modular form belonging to the $\Gamma_{\psi(15)}$ $(-2)$ Dimension:

\begin{equation}
	v=-2\pi i\frac{z_1}{z_0}\frac{\tau}{(\omega,d\omega)}=\left(\frac{2\pi}{\omega_2}\right)^2(1-3q^2-9q^4-3q^6-21q^8-\cdots),
\end{equation}

which is opposite to $W_ {15}$ character change, has its four simple zeros in the points that provide the branch points of that two-leaf surface. The power series then give the representation for $v^2$:

\begin{equation}
	v^2=z_0^4-10z_0^3z_1-13z_0^2z_1^2+10z_0z_1^3+z_1^4,
\end{equation}

with which the branching form is won. As a functional system of the transformation polygon $T_{15}$ we have:

\begin{equation}
	\tau(\omega)=\frac{z_0}{z_1},\quad\sigma(\omega)=\frac{v}{z_1^2}=-\frac{2\pi i}{z_0z_1}\frac{d\tau}{(\omega,d\omega)},
\end{equation}

where $\sigma$ in $\tau$ is represented by the square root:

\begin{equation}
	\sigma = \sqrt{\tau^4-10\tau^3-13\tau^2+10\tau+1}
\end{equation}

so that here we have an elliptic structure of the absolute invariant $\frac{13^3 37^3}{2^6 3^7 5^4}$.

To represent $ J $ as a rational function of $\sigma$ and $\tau$, we use the $\tau$ function introduced in (6) p.392 in the fifth-degree transformation, denoted by $\tau_5$ may, and in which $J$ is in the form (1.3) p.393. It is sufficient to represent the quadrivalent function $\tau_5$ in $\sigma$ and $\tau$ on $T_{15}$. If $\tau_5$ is transformed into $\tau'_5$ by $W_{15}$, you get:

\begin{equation}
  \tau_5=125\sqrt[4]{\frac{\Delta_5}{\Delta_1}},\quad
  \tau'_5=\sqrt[4]{\frac{\Delta_3}{\Delta_{15}}},\quad
\tau'_5-\tau_5=\frac{\sqrt[4]{\Delta_1\Delta_3}-125\sqrt[4]{\Delta_5\Delta_{15}}}{\sqrt[4]{\Delta_1\Delta_{15}}}
\end{equation}

The counter to the right is now a whole $(-6)^{ter}$ dimension of $K_ {15}$, which is opposite to $W_{15}$ character changes and thus contains the factor $v$. As a result, zero points of the total order 2 on $K_{15}$ done, so that even those of the total order 4 remain. A zero point of first order lies in the peak $i\infty$, so that the order 3 remains. Since this order is an integer, in the cycle $\pm1/3$, where our expression surely disappears, there must be at least one first-order zero point. Two more zeros of this order have to be determined. This consideration corresponds to the approach:

\[
  \sqrt[4]{\Delta_1\Delta_3}-125\sqrt[4]{\Delta_5\Delta_{15}}=v z_0 z_1(az_0^2+bz_0z_1+cz_1^2)
\]
The power series result in:

\begin{equation}
  \sqrt[4]{\Delta_1\Delta_3}-125\sqrt[4]{\Delta_5\Delta_{15}}=v z_0 z_1(z_0^2-4z_0z_1-z_1^2)
\end{equation}

To test this result, use (4) to compute the expression of $(\sqrt[4]{\Delta_1\Delta_3}+125\sqrt[4]{\Delta_5\Delta_{15}})^2$ in $z_0$, $z_1$, where the square of a homogeneous sixth degree function must result. This is confirmed, you can find by extracting the square root:

\begin{equation}
  \sqrt[4]{\Delta_1\Delta_3}+125\sqrt[4]{\Delta_5\Delta_{15}}=z_0 z_1(z_0^4-9z_0^3z_1-9z_0z_1^3-z_1^4).
\end{equation}

From (7) and (8) one now easily reaches the goal: The sought expression of the home Grade 5 occurring function $\tau_5$ in the current $\sigma$ and $\tau$ is:

\begin{equation}
  \tau_5=\frac{\tau^4-9\tau^3-9\tau-1-\sigma(\tau^2-4\tau-1)}{2\tau}.
\end{equation}
\end{document}