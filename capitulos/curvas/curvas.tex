\documentclass[../../tesis_maestria]{subfiles}
\begin{document}
\section{Curvas Algebraicas}

\subsection{Variedades Afines}
En esta sección revisamos las herramientas de geometría algebraica. Repasamos las definiciones básicas de variedades y sus morfismos. Después nos enfocamos en curvas para enunciar sus propiedades que usaremos en esta tesis. Luego estudiamos dos casos particulares: curvas elítpicas sobre $\CC$ y las curvas modulares $X_0(N)$. En esta sección fijamos un campo $K$ que sea algebraicamente cerrado, e.g. $K=\CC$ ó $K=\ol{\QQ}$.

Fijamos la siguiente notación: para $n>0$ y $K$ algebraicamente cerrado,
\begin{itemize}
	\item $\AAA^n_K:=\{(a_1,\ldots,a_n)\mid a_i\in K\}=K^n$ es el espacio afín con la topología de Zariski. A los puntos de $\AAA^n_K$ los denotamos por $a=(a_1,\ldots,a_n)$.
	\item $K[x]:=K[x_1,\ldots,x_n]$ el anillo de polinomios en $n$ variables con coeficientes en $K$. Si aparecen varios valores de $n$ a la vez, usaremos la notación completa.
	\item $\Ff(X,K):=\{f:X\ra K\}$ el anillo\footnote{Las operaciones de $\Ff(\AAA^n_K,K)$ son las inducidas por las operaciones de $K$, i.e. $(f+g)(a):=f(a)+g(a)$ y $(fg)(a):=f(a)g(a)$ donde $x=(a_1,\ldots,a_n)$} de funciones de $X$ a $K$. En particular vamos a estudiar $\Ff(\AAA^n_K,K)$ y sus subanillos.
\end{itemize}

\begin{nota}
	La topología de Zariski está definido por la siguiente base de cerrados: para todo ideal $I\subseteq K[x]$ define
	\[
		\VV_K(I):=\{a\in\mathbb{A}^n_K\mid f(a)=0\,\;\;\forall f\in I\}.
	\]
	Es decir, la topología de Zariski está definida por la base $\{\AAA^n_K-\VV_K(I)\}_I$. La asignación $I\mapsto\VV_K(I)$ nos da un mapeo de los ideales de $K[x]$ a los conjuntos cerrados de $\AAA^n_K$. Existe una operación (casi) inversa: para todo subconjunto $X\subseteq\AAA^n_K$ cerrado define
	\[
		\II_K(X):=\{f\in K[x]\mid f(a)=0\,\;\; \forall a\in X\}.
	\]
	Por el Nullstellensatz de Hillbert\footnote{La versión del Nullstellensatz que estamos usando es: sea $I$ un ideal de $K[x]$ con $K$ algebraicamente cerrado, entonces si $f\in K[x]$ es tal que $f(a)=0$ para toda $a\in\VV_K(I)$, entonces existe un natural $r>0$ tal que $f^r\in I$. Véase por ejemplo el capítulo IX, \S1 de \cite{LangA} o el teorema 25 de \S14 de \cite{MatsumuraCA} para otra formulación.}, tenemos que $\II_K(\VV_K(I))=\sqrt{I}$ donde $\sqrt{I}:=\{f\in K[x]\mid f^m\in I,\; m\gg 0\}$ es el radical del ideal $I\subseteq K[x]$. Conversamente, consideraciones estrictamente topológicas nos dan $\VV_K(\II_K(X))=\ol{X}$, la cerradura topológica de $X$ en $\AAA^n_K$. Esto nos da una biyección entre los ideales radicales de $K[x]$ (i.e. ideales $I$ tales que $\sqrt{I}=I$) y los conjuntos cerrados de $\AAA^n_K$. Véase \S1, capítulo 1 de \cite{HartshorneAG} para más detalles.
\end{nota}

\begin{defin}
	A los conjuntos cerrados $X\subseteq\AAA^n_K$, de la forma $X=\VV_K(I)$ para algún ideal $I\subseteq K[x]$, los llamamos \emph{conjuntos algebráicos afines} sobre $K$. A cada $X$ le asociamos su \emph{anillo de coordenadas} definido como el cociente
	\[
		K[X]:=\frac{K[x]}{\II_K(X)}
	\]
	que es una $K-$álgebra finitamente generada por el teorema de la base de Hilbert.\footnote{Simplemente dice, en lenguaje moderno, que si $A$ es un anillo noetheriano, entonces $A[x]$ es noetheriano. Con inducción sobre $n$ se prueba que $A[x_1,\ldots,x_n]$ es noetheriano. Por lo tanto todo cociente es noetheriano, en particular $K[X]$ para algún conjunto algebraico afín $X\subseteq\AAA^n_K$} Si además $I$ es primo, decimos que $X$ es una \emph{variedad afín} sobre $K$ y en este caso $K[X]$ es un dominio entero. Extendemos la definición de variedad a cualquier subconjunto abierto $Y\subseteq X$ definiendo $\II_K(Y):=\II_K(X)$ y $K[Y]:= K[X]$.
\end{defin}

\begin{nota}
	 En términos de la topología de Zariski, ser variedad afín es equivalente a ser un conjunto irreducible, es decir: Un subconjunto $Y$ de un espacio topológico $X$ es \emph{irreducible} si no existen subconjuntos propios $Y_1,Y_2\subsetneq Y$, cerrados en la topología de subespacio de $Y$, tales que $Y=Y_1\cup Y_2$. Véase por ejemplo el corolario 1.4 del capítulo I de \cite{HartshorneAG}).
\end{nota}

La asignación $X\mapsto K[X]$ le asocia una $K-$álgebra entera finitamente generada a cada variedad afín. Como en el caso de $\VV_K(I)$, también hay un inverso a esta construcción. Si $A$ es cualquier $K-$álgebra entera finitamente generada determina una variedad afín sobre $K$ con anillo de coordenadas $A$. En efecto, existe un homomorfismo sobreyectivo $K[x]\ra A$ y así $A\cong K[x]/I$ donde $I$ es primo porque $A$ es dominio, en particular $I$ es radical y así $\II_K(\VV_K(I))=\sqrt{I}=I$. Por lo tanto se definimos $X=\VV_K(I)$, obtenemos una variedad afín con anillo de coordenadas
\[
	K[X]=\frac{K[x]}{\II_K(X)}=\frac{K[x]}{I}=A.
\]
Por lo tanto podemos pensar a $K[X]$ como una realización algebraica de la variedad $X$.

Podemos interpretar a $K[X]$ como funciones de $X$ en $K$, i.e. como un subanillo de $\Ff(X,K)$. Para precisar esto primero vemos el caso $X=\AAA^n_K$. El anillo de polinomios $K[x]$ se puede identificar con el subanillo de $\Ff(\AAA^n_K,K)$ definido por
\begin{equation}\label{eq:funciones-polinomiales}
	\Big\{a\mapsto f(a)\Big| f\in K[x]\Big\}\subset\Ff(\AAA^n_K,K).
\end{equation}
Esto es posible porque cuando $K$ es infinito (e.g. cuando $K$ es algebraicamente cerrado), entonces la función $a\mapsto f(a)$ es la función $a\mapsto0$ si y solo si $f=0$. A las funciones en \eqref{eq:funciones-polinomiales} se les llaman \emph{funciones polinomiales} y lo denotamos por $K[\AAA^n_K]$. También es costumbre identificar a $K$ con el subanillo de funciones constantes en $\Ff(\AAA^n_K,K)$. Similarmente con las funciones polinomiales, el anillo de coordenadas $K[X]$ de un conjunto algebraico afín $X\subseteq\AAA^n_K$ se puede identificar con un subanillo $\Ff(X,K)$. En efecto, $f,g\in K[x]$ determinan la misma función en $\Ff(X,K)$ si y solo si para todo $a\in X$ tenemos que $(f-g)(a)=f(a)-g(a)=0$, i.e. que $f-g\in\II_K(X)$. Por lo tanto podemos identificar $K[X]=K[x]/\II_K(X)$ con el subanillo de funciones polinomiales en $\Ff(X,K)$, es decir que son restricción de funciones en $K[\AAA^n_K]$.

El anillo de coordenadas de un conjunto algebraico afín también se usa para definir \emph{dimensión} de $X$: es el número natural definido por la dimensión de Krull de $K[X]$, i.e.
\[
	\dim_{\mathrm{Krull}}K[X]:=\sup\{N\in\NN\mid \p_0\subsetneq\p_1\subsetneq\cdots\subsetneq\p_N\;\;\text{es una cadena de ideales primos}\}.
\]
En el caso particular de dimensión 1, $X\subseteq\mathbb{A}^n(K)$ es de dimensión 1 si y solo si $X=\VV(f)$ donde $f\in K[x]$ irreducible no constante (véase la proposición 1.13 de \S I.1 en \cite{HartshorneAG} para una prueba).

Ahora estudiamos una generalización de funciones polinomiales.

\begin{defin}
	Sea $X$ una variedad sobre $K$ y $a\in X$. Decimos que la función $f\in\Ff(X,K)$ es \emph{regular} en $a$ si existe una vecindad abierta $U\subseteq X$  de $a$ y polinomios $g,h\in K[x]$  tales que
\[
	f(a')=\frac{g(a')}{h(a')},\quad h(a')\neq0 \qquad\forall a'\in U.
\]
Si $f$ es regular en todos los puntos de un abierto $V\subseteq X$ (e.g. $V=X$), decimos que $f$ es regular sobre $V$. Al conjunto de funciones racionales regulares sobre $V$ lo denotamos $\Oo(V)$.
\end{defin}

\begin{nota}
En palabras, $f$ es regular en $a$ si localmente está bien definida como cociente de polinomios. Observa que la expresión $f=g/h$ no es única al menos de que $K[X]$ sea un dominio de factorización única, donde toda expresión se puede llevar a una expresión única al imponer la condición de que $g$ y $h$ sean primos relativos. Además, si $f$ es regular entonces es continua si le transferimos la topología de Zariski de $\AAA^1(K)$ a $K$. (cf. lema 3.1 del capítulo I de \cite{HartshorneAG}). Esto nos va a ayudar para definir morfismos entre variedades.
\end{nota}

Como el valor de una función regular en un punto $a\in X$ solamente depende de la información local alrededor de $a$, podemos estudiar los gérmenes de funciones regulares alrededor de $a$. Más precisamente, consideremos las parejas $(U,f)$ donde $f$ es regular sobre $U\subseteq X$, una vecindad abierta de $a$. A este conjunto de parejas lo denotamos:
\[
	\mathrm{Reg}_{a,X}:=\{(U,f)\mid a\in U\;\text{es abierto}, f\in\Ff(X,K)\;\text{es regular sobre}\; U\}.
\]
Esto no es notación estándar.

Decimos que dos parejas $(U,f)$ y $(V,g)$ son equivalentes si $f|_{U\cap V}=g|_{U\cap V}$.\footnote{El hecho que esto es efectivamente una relación de equivalencia se sigue de las siguientes dos observaciones: las funciones regulares son continuas y todo subconjunto abierto de $X$ es denso (gracias a que $X$ es irreducible como espacio topológico). En efecto si dos funciones regulares $f$ y $g$ coinciden en un abierto $U$, entonces son iguales porque el conjunto donde coinciden $W=\{a'\in X\mid f(a')-g(a')=0\}$ es cerrado y contiene a $U$, un abierto denso, por lo tanto $W=X$ o equivalentemente $f=g$.}

\begin{defin}
Con la notación anterior, una clase de equivalencia $[U,f]$ en $\mathrm{Reg}_{a,X}$ se llama un \emph{gérmen} de funciones regulares alrededor de $a$. Al anillo de gérmenes de funciones regulares alrededor de $a$ lo denotamos por $\Oo_{a,X}$ y se llama \emph{el anillo local} de $a$ en $X$.
\end{defin}

\begin{nota}
Observe que $\Oo_{a,X}$ es un anillo local con ideal maximal $\m_{a,X}=\{[U,f]\mid f(a)= 0\}$. Claramente $\m_{a,X}$ es un ideal. Para ver que es local, solamente debemos probar que cualquier elemento afuera de $\m_{a,X}$ es una unidad.\footnote{Aquí estamos usando la equivalencia entre anillos locales y anillos tales que sus elementos no invertibles forman un ideal (necesariamente maximal). Véase la proposición 1.6 de \cite{AtiyahITCA}.}. Sea $[U,f]\in\Oo_{a,X}$ tal que $f(a)\neq0$. Como $f$ es regular en $a$, es localmente un cociente de polinomios, es decir existe una vecindad $U'$ de $a$ tal que $f(a')=g(a')/h(a')$ para todo $a'\in U'$. Además, como $f(a)\neq 0$ implica que $g(a)\neq 0$, existe otra vecindad $U''$ de $a$ tal que $g(a'')\neq 0$ para toda $a''\in U''$. Esto quiere decir que el cociente $1/f=h/g$ está bien definido sobre la vecindad $U'\cap U''$ de $a$. Por lo tanto $[U'\cap U'',1/f]\in\Oo_{X,a}$ y es el inverso de $[U,f]$. 
\end{nota}

Ahora quitamos nuestra restricción a un punto $a$ y consideramos parejas $(U,f_U)$ donde $U\subseteq X$ es un abierto arbitrario y $f_U:U\ra K$ es regular. Similarmente al caso anterior, a este conjunto lo denotamos por $\mathrm{Reg}_X$. Decimos que dos parejas $(U,f_U),(V,f_V)\in\mathrm{Reg}_X$ son equivalentes si $f_U|_{U\cap V}=f_V|_{U\cap V}$. De esta manera:

\begin{defin}
Sea $X$ una variedad sobre $K$, al conjunto de clases de equivalencia, definidas en el párrafo anterior, lo denotamos por $K(X)$ y lo llamamos el \emph{campo de funciones de} $X$. Sus elementos se llaman \emph{funciones racionales} de $X$ y los denotamos por $f=[U,f_U]$, en este caso decimos que la pareja $(U,f_U)$ representa a la función racional.
\end{defin}

\begin{nota}
$K(X)$ es un campo gracias al mismo argumento que usamos para probar que $\Oo_{a,X}$ es un anillo local. Observa que para cada $a\in X$ tenemos contenciones naturales
\[
	K\inc\Oo(X)\inc\Oo_{a,X}\inc K(X)
\]
definidos por $f\mapsto[U,f]\mapsto[U,f]$ donde $U$ es una vecindad de $a$. Por lo tanto $K(X)$ es una $K-$álgebra. 
\end{nota}

El siguiente teorema, que tomamos de \cite{HartshorneAG}, relaciona la manera intrínseca de definir $\Oo(X),\Oo_{a,X}$ y $K(X)$ con el anillo de coordenadas de $X$ que depende de las ecuaciones que lo definen:

\begin{thm}\label{thm:prop-coord-X}
	Sea $X\subseteq\AAA^n_K$ una variedad algebraica afín. Entonces:
	\begin{enumerate}[label=(\roman*)]
		\item $K[X]\cong\Oo(X)$.
		\item $K(X)\cong\mathrm{Frac}(K[X])$, el campo de cocientes de $K[X]$.
		\item La función
		\[
			a \mapsto \m_a:=\{f\in K[X]\mid f(P)=0\}
		\]
		es una biyección entre los puntos de $X$ y los ideales maximales de $K[X]$.
		\item Para todo $a\in X$, tenemos que $\Oo_{a,X}\cong K[X]_{\m_a}$, la localización de $K[X]$ con respecto del ideal maximal $\m_a$.
		\item\label{dim-local-ring} $\dim_{\mathrm{Krull}}\Oo_{a,X}=\dim_{\mathrm{Krull}}K[X]=\dim X$.
	\end{enumerate}
\end{thm}

\begin{nota}
	Por (\emph{ii}), tenemos que $K(X)$ es una extensión finitamente generada de $K$ de grado de trascendencia $\dim X$ ya que este número coincide con el grado de trascendencia de $\mathrm{Frac}(K[X])$ sobre $K$ \cite[\S14]{MatsumuraCA}.
\end{nota}

El teorema anterior nos permite adaptar la definición usual de suavidad en cálculo para variedades algebraicas sobre $K$ de manera intrínseca, es decir que no depende de las ecuaciones que definen a la variedad. Recuerde que cada punto $a\in X$ tiene asociado un anillo local noetheriano $\Oo=\Oo_{a,X}$ con ideal maximal $\m=\m_{a,X}$. El campo residual $k=\Oo_{a,X}/\m_{a,X}$ actúa sobre $\m/\m^2$ por multiplicación escalar, i.e. $\m/\m^2$ es un $k-$espacio vectorial. Con esto definimos:

\begin{defin}
	Sea $X$ una variedad sobre $K$ y $a\in X$ un punto. Decimos que $X$ es \emph{suave} en $a$ si su anillo local $\Oo$ es un \emph{anillo local regular}, i.e. $\dim\Oo=\dim_k \m/\m^2$. Decimos que $X$ es \emph{suave} si es suave en todos sus puntos.
\end{defin}

\begin{nota}
	Esta definición de suavidad coincide con la noción clásica de suavidad definida por el Jacobiano. Más precisamente, sea $X\subseteq\AAA^n_K$ es una variedad afín cuyo ideal es generado por $f_1,\ldots,f_m\in K[x]$. Entonces $X$ es suave en un punto $a\in X$ si el rango de la matriz Jacobiana
	\[
		\begin{pmatrix}
		\tfrac{\partial f_1}{\partial x_1}(a) & \cdots & \tfrac{\partial f_1}{\partial x_n}(a) \\
		\vdots & \ddots & \vdots \\
		\tfrac{\partial f_m}{\partial x_1}(a) & \cdots & \tfrac{\partial f_m}{\partial x_n}(a)
		\end{pmatrix}
	\]
	es igual a $n-\dim X$. Véase el teorema 5.1 del primer capítulo de \cite{HartshorneAG} para una prueba de esto.
\end{nota}

Ahora definimos morfismos entre variedades y veremos que los anillos $\Oo(X)$, $\Oo_{P,X}$ y $K(X)$ son invariantes asociados a la variedad $X$.

\begin{defin}
	Sean $X$ y $Y$ variedades sobre $K$ Decimos que una función $\varphi:X\ra Y$ es un \emph{morfismo} de variedades sobre $K$ si es continua con respecto de las topologías de Zariski y si para toda $V\subseteq Y$ abierto y para toda $f\in\Oo(V)$, entonces $f\circ\varphi\in\Oo(\varphi^{-1}(V))$.
\end{defin}

\begin{nota}
	La composición de morfismos es morfismo entonces, dos variedades $X$ y $Y$ sobre $K$ son isomorfos si existen morfismos $\varphi:X\ra Y$ y $\psi: Y\ra X$ tales que $\psi\circ\varphi=\Id_X$ y $\varphi\circ\psi=\Id_Y$.
\end{nota}

Cada morfismo $\varphi: X\ra Y$ determina un morfismo de $K-$álgebras:
\begin{equation}\label{eq:morf-inducido-coord}
	\varphi^*:K[Y]\lra K[X] \quad\text{definido por}\quad f\mapsto\varphi\circ f.
\end{equation}
La asignación $\varphi\mapsto\varphi^*$ es una biyección entre el conjunto de morfismos $X\ra Y$ y el conjunto de homomorfismos de $K-$álgebras $K[Y]\ra K[X]$. En efecto, su inverso se puede construir de la siguiente manera:

Sea $f:K[Y]\ra K[X]$ un homomorfismo de $K-$álgebras. Bajo $f$, las funciones coordenadas $x_i\in K[Y]$ corresponden a $x'_i:=f(x_i)\in K[X]$. La función $f_*:X\ra Y$ definida por $a=(a_1,\ldots,a_n)\mapsto(x'_1(a),\ldots,x'_m(a))$ es el inverso de $\varphi\mapsto \varphi^*$. Para más detalles, véase la prueba de la proposición 3.5 del capítulo I de \cite{HartshorneAG}.

Esta biyección nos sugiere que $X\mapsto K[X]$ es una equivalencia de categorías. Más precisamente, denotamos por $\mathbf{VarAfin_K}$ a la categoría de variedades sobre $K$ con los morfismos sobre $K$ y denotamos por $\mathbf{K}$-$\mathbf{alg.fg}$ a la categoría de $K-$álgebras finitamente generadas sin divisores de cero, junto con homomorfismos de $K-$álgebras. Entonces tenemos:

\begin{thm}\label{thm:equiv-afin-k-alg}
	La asignación $X\mapsto K[X]$ junto con $(X\morf{\varphi}Y)\mapsto(K[Y]\morf{\varphi^*}K[X])$, es una equivalencia (contravariante) de categorías $\mathbf{VarAfin_K} \morf{\sim}\mathbf{K}$-$\mathbf{alg.fg}$. En particular, si $X$ y $Y$ son variedades afines, entonces:
	\[
		X\cong Y\quad\iff\quad K[X]\cong K[Y].
	\]
\end{thm}

Un resultado muy similar se puede probar con $K(X)$ en lugar de $K[X]$, pero requiere otro tipo de morfismo entre variedades cuya construcción es muy similar a la construcción de $K(X)$. Para definirlo sean $X$ y $Y$ variedades sobre $K$ y consideramos parejas $(U,\varphi_U)$ donde $U\subseteq X$ es un abierto y $\varphi_U:U\ra Y$ es un morfismo de variedades sobre $K$. Análogamente a la construcción de $K(X)$, decimos que $(U,\varphi_U)\sim(V,\varphi_V)$ si y solo si $\varphi_U|_{U\cap V}=\varphi_V|_{U\cap V}$.

\begin{defin}
Sean $X$ y $Y$ variedades sobre $K$. Con la notación de arriba, una clase de equivalencia $\varphi=[U,\varphi_U]$ se llama un \emph{mapeo racional} de $X$ a $Y$ y se denota por $\varphi:X\ra Y$. Decimos que $\varphi$ es \emph{dominante} si la imagen de $\varphi_U$ es densa en $Y$.\footnote{Esta definición no depende de la elección de pareja $(U,\varphi_U)$ para representar a $\varphi$ ya que si dos morfismos de $X$ a $Y$ coinciden sobre un abierto, son iguales (véase el lema 4.1 del capítulo I de \cite{HartshorneAG}).}
\end{defin}

\begin{nota}
La composicion de dos mapeos racionales $\varphi:X\ra Y$ y $\psi:Y\ra Z$ representados por $(U,\varphi_U)$ y $(V,\psi_V)$ respectivamente se define como el mapeo racional $\psi\circ \varphi: X\ra Z$ representado por $(U,\psi_V|_{\varphi_U(U)\cap V}\circ\varphi_U)$. Si $\varphi$ y $\psi$ son dominantes, entonces $\psi\circ \varphi$ también lo es porque la imagen de $\psi_V|_{\varphi_U(U)\cap V}\circ\varphi_U$ es densa en $Z$. Por lo tanto podemos considerar a la categoría de variedades sobre $K$ con mapeos racionales dominantes que denotamos por $\mathbf{VarAfin_K^{rac}}$. Si existen mapeos racionales dominantes $\varphi:X\ra Y$ y $\psi:Y\ra X$ tales que $\psi\circ\varphi=\Id_X$ y $\varphi\circ\psi=\Id_Y$, decimos que $X$ y $Y$ son \emph{birracionalmente equivalentes} y lo denotamos por $X \approx Y$.
\end{nota}

Para relacionar esta contrucción con el campo de funciones $K(X)$, tomamos un mapeo racional dominante $\varphi:X\ra Y$ representado por $(U,\varphi_U)$ y vamos a construir un homomorfismo $K(Y)\ra K(X)$. Sea $g\in K(Y)$ representado por la pareja $(V,g_V)$ donde $g_V:V\ra K$ es regular. Primero observa que como $\varphi_U(U)\subset Y$ es denso, $\varphi_U(U)\cap V$ es no vacío, por lo tanto $\varphi^{-1}_U(V)$ es no vacío y abierto, porque $\varphi_U$ es continua por ser morfismo. Otra vez por la definición de morfismo, la composición $g_V\circ\varphi_U:\varphi_U^{-1}(V)\ra K$ es regular sobre un abierto de $X$ y por lo tanto podemos definir $g\mapsto[\varphi^{-1}(V),g_V\circ\varphi_U]\in K(X)$. Esta función es un morfismo de $K-$álgebras.

Similarmente a la construcción $\varphi\mapsto\varphi^*$ de \eqref{eq:morf-inducido-coord}, la asignación de un morfismo de $K-$álgebras a un mapeo racional dominante es una biyección entre el conjunto de morfismos de $K-$álgebras $K(Y)\ra K(X)$ y el conjunto de mapeos birracionales dominantes $X\ra Y$. Por lo tanto la asignación $X\mapsto K(X)$ es un funtor a la categoría de extensiones de $K$ finitamente generadas, que denotamos por $\mathbf{K}$-$\mathbf{ext,fg}$. De esta manera tenemos un resultado similar al teorema \ref{thm:equiv-afin-k-alg}:

\begin{thm}
	El funtor $X\mapsto K(X)$ es una equivalencia de categorías $\mathbf{VarAfin_K} \morf{\sim}\mathbf{K}$-$\mathbf{ext.fg}$. En particular dos variedades $X$ y $Y$ son birracionalmente equivalentes si y solo si $K(X)$ y $K(Y)$ son isomorfas como $K-$álgebras, i.e.
	\[
		X\approx Y\quad\iff\quad K(X)\cong K(Y).
	\]
\end{thm}


------

Por otro lado, el espacio proyectivo $\PP^n(K)$ se define como el espacio cociente de la acción $K^*\act\mathbb{A}^n(K)-\{0\}$ que escala, i.e. consiste de clases de equivalencia $[x_0;\cdots;x_n]$ donde no todas las $x_i$'s son cero y tales que $[x_0;\cdots;x_n]=[x'_0;\cdots;x'_n]$ si y solo si existe un escalar $\la\in K^*$ tal que $\la x_i=x'_i$ para toda $i$. Ahora, un polinomio $f\in K[x_0,\ldots,x_n]$ no necesariamente define una función $\PP^n(K)\ra K$, pero cuando $f$ es \emph{homogéneo}\footnote{Decimos que un polinomio $f\in K[x_0,\ldots,x_n]$ es homogéneo de grado $d$ si todos sus monomios son de grado $d$, o equivalentemente $f(\la x_0,\ldots,\la x_n)=\la^d f(x_0,\ldots,f_n)$ para toda $\la\in K$.}, podemos definir bien cuándo $f$ se anula, i.e. $f[x_0,\ldots,x_n]=0\iff f(x_0,\ldots,x_n)=0$.

Por lo tanto decimos que $X\subset\PP^n(K)$ es un \emph{conjunto algebraico proyectivo} si existe un \emph{ideal homogéneo} $I\subseteq K[x_0,\ldots,x_n]$, i.e. generado por polinomios homogéneos, tal que $X$ es igual a
\[
	\VV_{\PP^n(K)}(I):=\{P\in\PP^n(K)\mid f(P)=0,\;\;\forall f\in K[x_0,\ldots,x_n]\;\;\text{homogéneo}\}.
\]
Análogamente definimos para $X$ un conjunto proyectivo su ideal asociado
\[
	\II_{\PP^n(K)}(X):=\gen{f\in K[x_0,\ldots,x_n]\;\;\text{homogéneo}\mid f(P)=0,\;\;\forall P\in X}.
\]
Además decimos que $X$ es una \emph{variedad proyectiva} si $\II_{\PP^n(K)}(X)$ es un ideal primo de $K[x_0,\ldots,x_n]$. Similarmente al caso afín, $\PP^n(K)$ tiene una topología de Zariski generado por los complementos de los conjuntos $\VV_{\PP^n(K)}(I)$.

Una ventaja de trabajar en $\PP^n(K)$ es que está cubierto por espacios afines $\AA^n(K)$. Más precisamente consideremos el conjunto algebraico proyectivo $\VV_{\PP^n(K)}(x_i)$, i.e. el conjunto de puntos $[x_0,\ldots,x_n]$ tales que $x_i=0$, su complemento $U_i:=\PP^n(K)-\VV_{\PP^n(K)}(x_i)$ es un conjunto abierto en la topología de Zariski y además la función
\[
	\varphi_i:U_i\lra\mathbb{A}^n(K)\quad\text{definida por}\quad [x_0,\ldots,x_n]\mapsto \paren{\frac{x_0}{x_i},\ldots,\frac{x_{i-1}}{x_i},\frac{x_{i+1}}{x_i},\ldots,\frac{x_n}{x_i}}
\]
es un homeomorfismo donde $U_i\subset\PP^n(K)$ tiene la topología inducida por la topología de Zariski del espacio proyectivo (cf. la proposición 2.2 de \S2 del capítulo I de \cite{HartshorneAG}). Claramente $\{U_i\}_{i=0,\ldots,0}$ es una cubierta abierta de $\PP^n(K)$, es decir $\PP^n(K)=\cup U_i$, entonces $\{X\cap U_i\}$ es una cubierta de $X$, donde cada $X\cap U_i$ es homeomorfo a su imagen en $\mathbb{A}^n(K)$ bajo $\varphi_i$. Como $X\cap U_i$ es cerrado en $U_i$, su imagen es cerrado, i.e. es una conjunto algebraico que denotamos $X_i:=\varphi_i(X\cap U_i)$. Las parejas $(X\cap U_i,\varphi)$ se llaman cartas afines de la variedad $X$.

Esto nos permite asociarle variedades afines a una variedad proyectiva. El converso también es cierto, es decir a podemos asociar una variedad proyectiva a cualquier variedad afín. Sea $X\subseteq\mathbb{A}^n(K)$ y toma la cerradura topológica de $\varphi_i^{-1}(X)$ en $\PP^n(K)$; a este cerrado lo denotamos por $\ol{X}$ y lo llamamos la \emph{proyectivización} de $X$. Con esta notación, podemos precisar la relación entre variedades afines y proyectivas con:

\begin{prop}
Para toda variedad proyectiva $X\subseteq\PP^n(K)$ los conjuntos algebraicos afines $X_i\neq\emptyset$ son variedades cuya proyectivizaciones son $X$, i.e. $\ol{X_i}=X$.	Por el contrario, para toda variedad afín $X\subset\mathbb{A}^n(K)$ existe una variedad proyectiva $\ol{X}\subseteq\PP^n(K)$, tal que $X=(\ol{X})_i$ para alguna $i$.
\end{prop}

\begin{proof}
	Esto es la proposición 2.3 y su corolario en \S2 del capítulo I de \cite{HartshorneAG}.
\end{proof}

\begin{nota}
	Los diferentes anillos de coordenadas $K[X_i]$ son isomorfas como $K-$álgebras y en general las diferentes $X_i$'s son birracionalmente equivalentes, entonces módulo equivalencia birracional, podemos denotar $X\cap\mathbb{A}^n(K):= X_i$. La proposición anterior se reescribe como $\ol{X\cap\mathbb{A}^n(K)}=X$ (resp. $\ol{X}\cap\mathbb{A}^n(K)=X$) cuando $X$ es una variedad proyectiva (resp. afín). La asociación $X\mapsto X\cap\mathbb{A}^n(K)$ nos permite definir la dimensión de una variedad proyectiva $X$ como la dimensión de $X\cap\mathbb{A}^n(K)$. Hacemos lo mismo con el campo de funciones, es decir definimos el campo de funciones de la variedad proyectiva $X$ como $K(X\cap\mathbb{A}^n(K))$; recuerde que si cambiamos de carta afín obtenemos $K-$álgebras isomorfas.
\end{nota}

Ahora, discutimos los morfismos entre variedades proyectivas. Sean $X,X\subseteq\PP^n(K)$ variedades proyectivas y $f_0,\ldots,f_n\in K(X)$







\subsection{El teorema de Riemann-Roch}

\begin{thm}\label{thm:riemann-roch}(Riemann-Roch)
	Sea $\Cc$ una curva definida sobre cualquier campo y sea $K$ su divisor canónico. Entonces existe un entero $g$, el género de $\Cc$ tal que para todo divisor $D\in\mathrm{Div}(\Cc)$ se cumple:
	\[
		\dim\Ll(D)-\dim\Ll(K-D)=\deg D-g+1.
	\]
\end{thm}

La prueba de Riemann-Roch se puede hacer con el teorema de dualidad de Serre, e.g. en el capítulo IV de \cite{HartshorneAG} o en \cite{SerreGAECDC}


\begin{cor} Con las hipótesis del teorema de Riemann-Roch tenemos:
\begin{enumerate}[label=(\roman*)]
\item $\dim\Ll(K)=g$.
\item $\deg K=2g-2$
\item $\deg D>2g-2\quad\then\quad \dim\Ll(D)=\deg D-g+1$
\end{enumerate}
\end{cor}

\subsection{Curvas modulares}

\subsubsection{El modelo racional de $X_0(N)$
}
\begin{thm}\label{thm:modelo-racional}
	Para toda curva modular $X_0(N)$ existe una única (módulo isomorfismo) curva proyectiva suave sobre $\QQ$, que denotamos por $X_0(N)_\QQ$, y un isomorfismo $\varphi_N:X_0(N)\ra X_0(N)_\QQ(\CC)$ sobre $\CC$ entre los $\CC-$puntos tal que el morfismo inducido en campo de funciones, $\varphi_N^*:\CC(X_0(N)_\QQ)\ra\CC(X_0(N))$, cumple que $\varphi_N^*(\QQ(X_0(N)_\QQ))=\QQ(j,j_N)$.
\end{thm}


\subsubsection{Espacios moduli}\label{sec:curvas-modulares}
En esta sección vemos como las curvas modulares parametrizan clases de isomorfismos entre curvas elípticas que preservan cierta información de torsión de las curvas elípticas.

El siguiente paso es estudiar la categoría de curvas elípticas con un subgrupo cíclico de orden 15 distinguido. Más precisamente, los objetos son parejas $(E',C')$ donde $E'/\CC$ es una curva elíptica con un subgrupo cíclico $C'$ de $E(\ol{\QQ})$ de orden $N=15$. Los isomorfismos de esta categoría son isomorfismos $\varphi:E'\ra E''$ tales que $\varphi(C')= C''$ que definen una relación de equivalencia sobre estas parejas que denotamos por $(E',C')\sim(E'',C'')$. Al conjunto de clases de equivalencias lo denotamos $S_0(N)=\{[E',C']\}$. Si queremos restringir las curvas elípticas a curvas sobre algún subcampo $K$ de $\CC$, denotamos por $S_0(N)(K)$ al conjunto de clases de equivalencias $[E',C']$ donde $E'$ está definida sobre $K$, el isomorfismo $E'\ra E''$ está definida sobre $K$ y $C'\subset E'(K)$. Observe que $S_0(N)(\CC)=S_0(N)$.

\begin{nota}
En general, podemos identificar $S_0(N)(K)$ con otro conjunto de clases de equivalencia. En este caso los objetos son isogenias $\varphi:E\ra E'$ definidas sobre $K$ con núcleo cíclico de orden $N$; a éstas se les llaman $N-$\emph{isogenias}. Un isomorfismo entre dos $N-$isogenias $\varphi_1:E_1\ra E'_1$ y $\varphi_2:E_2\ra E'_2$ es una pareja de isomorfismos $E_1\cong E_2$ y $E'_1\cong E'_2$, cada uno definido sobre $K$, tales que el siguiente diagrama conmuta:
\begin{equation}\label{eq:diagrama-iso-isogenias}
	\begin{tikzcd}
		E_1 \arrow[d,"\varphi_1"'] \arrow[r,"\sim"] & E_2 \arrow[d,"\varphi_2"]\\
		E'_1 \arrow[r,"\sim"] & E'_2
	\end{tikzcd}
\end{equation}
A la clase de isomorfismo de $N-$isogenias, lo denotamos $[E\morf{\varphi} E']$. Al conjunto de clases de isomorfismo de isogenias lo denotamos por $\mathrm{Isog}_N(K)$. De esta manera podemos identificar $S_0(N)(K)$ con $\mathrm{Isog}_N(K)$. Para esto consideramos la siguiente función:
\begin{equation}\label{def:phi-N-K}
	\Phi_{N,K}:\mathrm{Isog}_N(K)\lra S_0(N)(K)\quad\text{definido por}\quad [E\morf{\varphi} E']\mapsto [E,\ker\varphi]
\end{equation}
La función $\Phi_{N,K}$ es claramente bien definida por la conmutatividad del diagrama \eqref{eq:diagrama-iso-isogenias}.

Para construir su inverso, recuerde que para todo subgrupo finito $C$ de $E$, existe una única curva elíptica, denotada por $E/C$, y una isogenia $E\ra E/C$ con núcleo $C$ (cf. teorema \ref{thm:kernel-isogenias}), i.e. una $N-$isogenia\marginpar{\scriptsize{Este teorema es nuevo en la sección de curvase elipticas}}. Además, como $E$ está definida sobre $K$ y $C\subset E(K)$, la curva $E/C$ y la isogenia están definidas sobre $K$. Esta asignación sugiere que el inverso de $\Phi_{N,K}$ es la función $[E,C]\mapsto[E\ra E/C]$. En efecto, la unicidad de la curva $E/C$ garantiza que está bien definida y claramente es el inverso de $\Phi_{N,K}$. Por lo tanto $\Phi_{N,K}$ es una biyección.
\end{nota}

Ahora hacemos una primera reducción para estudiar $S_0(N)$. Gracias al teorema de uniformización (cf. el teorema \ref{thm:unif}), para toda curva elíptica $E/\CC$ existe un $\tau\in\HH$ tal que $E\cong \CC/(\tau\ZZ\oplus\ZZ)$ como grupos de Lie; denotamos $\Lambda_\tau:=\tau\ZZ\oplus\ZZ$ y  $E_\tau:=\CC/\Lambda_\tau$. Con esta notación tenemos el siguiente lema:

\begin{lema}\label{lema:espacio_moduli_Y_0(N)}
	Todo elemento $[E,C]\in S_0(N)$ es de la forma $[E_\tau,\gen{N^{-1}+\Lambda_\tau}]$ para alguna $\tau\in\HH$. Además, la función $\Psi_N:S_0(N)\ra Y_0(N)$ definido por $[E_\tau,\gen{N^{-1}+\Lambda_\tau}]\mapsto \tau\Gamma_0(N)\in Y_0(N)$ es una biyección.
\end{lema}

\begin{proof}
Sea $[E,C]\in S_0(N)$ y sea $Q\in C$ un generador, en particular $Q$ es de orden $N$. Por el teorema de uniformización $E_\tau\cong E$ para alguna $\tau$. Bajo este isomorfismo, $Q$ corresponde a un punto de $E_\tau$ que denotamos por
\[	
	Q=z_0+\Lambda_\tau\qquad(z_0\in\CC).
\]
Como $Q$ es de orden $N$, entonces $Nz_0\in\Lambda_\tau$ lo cual implica que existen $a,b\in\ZZ$ tales que $Nz_0=a\tau+b$. Como $\{1,\tau\}$ es una $\RR-$base de $\CC$, existen $\la,\mu\in\RR$ tales que $z_0=\la\tau+\mu$. Si igualamos ambas expresiones de $z_0$, obtenemos que $N\la=a$ y $N\mu=b$ y por lo tanto $\la,\mu\in\QQ$. De otra manera:
\[
	Q=\frac{\alpha\tau+\beta}{\gamma}+\Lambda_\tau\qquad(\alpha,\beta,\gamma\in\ZZ).
\]
Otra vez por el orden de $Q$, multiplicamos la ecuación anterior por $N$ y obtenemos: $N(\alpha\tau+\beta)/\gamma\in\Lambda_\tau$ y así $N\alpha/\gamma,N\beta/\gamma\in\ZZ$. Sin pérdida de generalidad podemos tomar $(\alpha,\beta,\gamma)=1$, entonces podemos concluir que $\gamma\mid N$. Por otro lado, si $\gamma<N$ entonces $\gamma Q=\alpha\tau+\beta+\Lambda_\tau=\Lambda_\tau$ lo cual contradice que $Q$ tiene orden $N$. Por lo tanto $\gamma=N$ y podemos asumir que existen $c,d\in\ZZ$ tales que
\[
	Q=\frac{c\tau+d}{N}+\Lambda_\tau\qquad(c,d,N)=1.
\]
Observe que si $t,t'\in\ZZ$ entonces la ecuación
\[
	\frac{(c+tN)\tau+(d+t'N)}{N}+\Lambda_\tau=
	\frac{c\tau+d}{N}+t\tau+t'+\Lambda_\tau=
	\frac{c\tau+d}{N}+\Lambda_\tau=Q
\]
implica que la elección de $c$ y $d$ depende solamente de sus clases módulo $N$.

Por las hipótesis sobre $c,d$ y $N$, existen $a,b,k\in\ZZ$ tales que $ad-bc+kN=1$ es decir, si denotamos
\[
	\sigma=\mat{a}{b}{c}{d}\in M_2(\ZZ),
\]
entonces bajo la proyección $\pi:M_2(\ZZ)\epi M_2(\ZZ/N\ZZ)$, tenemos que $\pi(\sigma)\in\SL_2(\ZZ/N\ZZ)$. Como la restricción $\pi:\SL_2(\ZZ)\ra\SL_2(\ZZ/N\ZZ)$ es sobreyectiva, toma $$\sigma'=\mat{a'}{b'}{c'}{d'}\in\SL_2(\ZZ)$$ tal que $\pi(\sigma')=\pi(\sigma)$. Por construcción, $c\equiv c'\Mod{N}$ y $d\equiv d'\Mod{N}$ entonces
$Q=(c'\tau+d')/N+\Lambda_{\tau}$.

Sea $\tau'\in\HH$ tal que
\begin{equation}\label{eq:def_de_matriz_sigma}
  \tau'=\sigma'\tau=\frac{a'\tau+b'}{c'\tau+d'}
\end{equation}
y denotamos al denominador por $m=c'\tau+d'$. Entonces $m\tau'=(a'\tau+b')$ y así
\begin{equation}\label{eq:reticulas_iguales}
	m\Lambda_{\tau'}=
	m(\tau'\ZZ\oplus\ZZ)=
	m\tau'\ZZ\oplus m\ZZ=
	(a'\tau+b')\ZZ\oplus(c'\tau+d')\ZZ.
\end{equation}

Es conocido que dos retículas $\omega_1\ZZ\oplus\omega_2\ZZ$ y $\omega'_1\ZZ\oplus\omega'_2\ZZ$, tales que $\Im(\omega_1/\omega_2),\Im(\omega'_1/\omega'_2)>0$, son iguales si $\omega_1/\omega_2,\omega'_1/\omega'_2\in\HH$ están en la misma órbita de la acción $\PSL_2(\ZZ)\act\HH$.\footnote{\label{foot:reticulas}Más precisamente, si $\mathcal{R}$ es el espacio de retículas, $\CC^*$ actúa por homotecias. Entonces $\omega_1\ZZ\oplus\omega_2\ZZ\mapsto \omega_1/\omega_2$ es una biyección $\mathcal{R}/\CC^*\ra\HH/\PSL_2(\ZZ)$ (cf. la proposición 3 de $\S2$ del capítulo VII de \cite{SerreACIA}). En particular tenemos que
\[
	\omega_1\ZZ\oplus\omega_2\ZZ=\omega'_1\ZZ\oplus\omega'_2\ZZ \quad\iff\quad
	\omega'_1=a\omega_1+b\omega_2,\;\;\omega'_2=c\omega_1+d\omega_2\quad\text{donde}\;\;\mat{a}{b}{c}{d}\in\SL_2(\ZZ)
\]} En este caso tenemos que $(a'\tau+b')\ZZ\oplus(c'\tau+d')\ZZ=\tau\ZZ\oplus\ZZ$ porque $(a'\tau+b')/(c'\tau+d')=\tau'=\sigma'(\tau)$ y así $\tau/1$ y $(a'\tau'+b')/(c'\tau'+d')$ están en la misma órbita. De \eqref{eq:reticulas_iguales} concluimos que $m\Lambda_{\tau'}=\Lambda_{\tau}$ y que
\[
	m\paren{\frac{1}{N}+\Lambda_{\tau'}}=\frac{c'\tau+d'}{N}+\Lambda_\tau=Q.
\]
Por lo tanto el homomorfismo $E_{\tau'}\ra E_\tau$ definido por $z+\Lambda_{\tau'}\mapsto mz+\Lambda_{\tau}$ es un isomorfismo\marginpar{\scriptsize{No sé si agregar una parte de retículas a la sección de curvas elípticas para justificar este isomorfismo}}. Si lo componemos con el isomorfismo $E_\tau\cong E$ obtenemos un isomorfismo $f:E_{\tau'}\ra E$ donde $f(N^{-1}+\Lambda_{\tau'})=Q$. De esta manera $f\big(\gen{N^{-1}+\Lambda_{\tau'}}\big)=\gen{Q}=C$. Concluimos que $[E,C]=[E_{\tau'},\gen{N^{-1}+\Lambda_{\tau'}}]$ para alguna $\tau'\in\HH$.

Ahora demostramos la segunda parte del lema. Tenemos que probar que la función $\Psi_N:[E_\tau,\gen{N^{-1}+\Lambda_\tau}]\mapsto\tau\Gamma_0(N)$ cumple tres cosas:
\begin{enumerate}[label=\emph{\roman*})]

  \item $\Psi_N$ está bien definida.
  
  \-\;\;Si $[E_\tau,\gen{N^{-1}+\Lambda_\tau}]=[E_{\tau'},\gen{N^{-1}+\Lambda_{\tau'}}]$, entonces $\CC/\Lambda_\tau\cong\CC/\Lambda_{\tau'}$ y por lo tanto existe un $m\in\CC^*$ tal que $m\Lambda_\tau=\Lambda_{\tau'}$ y tal que $m\gen{N^{-1}+\Lambda_\tau}=\gen{N^{-1}+\Lambda_{\tau'}}$ (véase la nota de pie \ref{foot:reticulas}). Como $\Lambda_\tau=\tau\ZZ\oplus\ZZ$, entonces la igualdad $m\Lambda_\tau=m\tau\ZZ\oplus m\ZZ=\tau'\ZZ\oplus\ZZ$ nos dice que existe un
\[
	\sigma=\mat{a}{b}{c}{d}\in\SL_2(\ZZ)
\]
tal que
\[
	m\tau=a\tau'+b\quad,\quad m=c\tau'+d
\]
o en particular $\sigma\tau'=\tau$; esto es otra vez por la nota de pie \ref{foot:reticulas}.

  \-\;\; Por otro lado sabemos que el isomorfismo $E_\tau\cong E_{\tau'}$ manda $N^{-1}+\Lambda_\tau$ en $\gen{N^{-1}+\Lambda_{\tau'}}$, es decir
\[
	m\paren{\frac{1}{N}+\Lambda_\tau}=\frac{c\tau'+d}{N}+\Lambda_{\tau'}=\frac{k}{N}+\Lambda_{\tau'}\quad (1\leq k< N)
\]
donde $(k,N)=1$ porque $k/N+\Lambda_{\tau'}$ es necesariamente de orden $N$. La ecuación anterior implica que
\[
	\frac{c}{N}\tau'+\frac{d-k}{N}\in\Lambda_{\tau'}\quad\then\quad N\mid c,\;\; N\mid d-k.
\]
En particular $c\equiv0\Mod{N}$. Adem\'as, si $\delta$ fuese un factor común de $N$ y $d$, entonces $\delta\mid d-k$ implica que $\delta\mid k$ y así $\delta\mid(N,k)=1$. Por lo tanto $(N,d)=1$ y así deducimos que $d\equiv1\Mod{N}$. Con esto concluimos que $\sigma\in\Gamma_0(N)$. Como $\sigma\tau'=\tau$, tenemos que $\tau\Gamma_0(N)=\tau'\Gamma_0(N)$ cuando $[E_\tau,\gen{N^{-1}+\Lambda_\tau}]=[E_{\tau'},\gen{N^{-1}+\Lambda_{\tau'}}]$ y por lo tanto la función $[E_\tau,\gen{N^{-1}+\Lambda_\tau}]\mapsto\tau\Gamma_0(N)$ está bien definida.

  \item $\Psi_N$ es inyectiva.
  
  \-\;\; Sean $\tau,\tau'\in\HH$ tales que $\tau\Gamma_0(N)=\tau'\Gamma_0(N)$, por ejemplo $\tau'=\sigma'\tau$ donde $\sigma'\in\Gamma_0(N)$ y es de la forma \eqref{eq:def_de_matriz_sigma}. De manera análoga a \eqref{eq:reticulas_iguales} y al párrafo que le sigue, concluimos que $m\Lambda_{\tau'}=\Lambda_{\tau}$, donde $m=c'\tau+d'$, y que
  \[
  	m\paren{\frac{1}{N}+\Lambda_{\tau'}}=\frac{c'\tau+d'}{N}+\Lambda_\tau.
  \]
De esta manera $E_\tau=\CC/\Lambda_\tau\cong\CC/\Lambda_{\tau'}=E_{\tau'}$ donde el isomorfismo está dado por $z+\Lambda_{\tau'}\mapsto mz+\Lambda_{\tau}$. Además, como $\sigma'\in\Gamma_0(N)$, entonces $N\mid c'$ y así $c'=Nc$ para alguna $c\in\ZZ$. Por lo tanto
\[
  m\paren{\frac{1}{N}+\Lambda_{\tau'}}=c\tau+\frac{d'}{N}+\Lambda_\tau=\frac{d'}{N}+\Lambda_\tau,
\]
donde, como $(N,d')=1$, $d'/N+\Lambda_\tau$ es un generador del subgrupo cíclico $\gen{N^{-1}+\Lambda_\tau}$. Por lo tanto el isomorfismo $z+\Lambda_{\tau'}\mapsto mz+\Lambda_{\tau}$ manda al subgrupo $\gen{\frac{1}{N}+\Lambda_{\tau'}}$ en el subgrupo $\gen{N^{-1}+\Lambda_\tau}$. Por lo tanto $[E_\tau,\gen{N^{-1}+\Lambda_\tau}]=[E_{\tau'},\gen{N^{-1}+\Lambda_{\tau'}}]$ y $\Psi_N$ es inyectiva.

  \item $\Psi_N$ es sobre.
  
  \-\;\; Esto es claro porque $\tau\Gamma_0(N)$ viene de la curva elíptica $E_\tau$ con subgrupo cíclico fijo $\gen{N^{-1}+\Lambda_\tau}$, i.e. $\Psi_N[E_\tau,\gen{N^{-1}+\Lambda_\tau}]=\tau\Gamma_0(N)$ y $\Psi_N$ es sobre.
\end{enumerate}
\end{proof}

\begin{nota}
	Recuerde que $S_0(N)$ se puede identificar con $\mathrm{Isog}_N(\CC)$, i.e. clases de isomorfismo de isogenias con núcleo cíclico de orden $N$. Las clases $[E_\tau,\gen{N^{-1}+\Lambda_\tau}]$ corresponden a la clase $[E_\tau\morf{\varphi} E_{N\tau}]$ donde la isogenia $\varphi$ es multiplicar por $N$. En efecto, el núcleo de $\varphi$ consiste de puntos $z+\Lambda_\tau$ tales que $Nz\in\Lambda_{N\tau}$, es decir $Nz=a+bN\tau$ para algunas $a,b\in\ZZ$. Como $\{1,\tau\}$ es $\RR-$base de $\CC$, sabemos que existen $\la,\mu\in\RR$ tales que $z=\la+\mu\tau$ y por lo tanto $N\la=a$ y $\mu=b$. Esto quiere decir que 
	\[
	z+\Lambda_\tau=\left(\frac{a}{N}+b\tau\right)+\Lambda_\tau=\frac{a}{N}+\Lambda_\tau\in\gen{N^{-1}+\Lambda_\tau}
	\]
y por lo tanto $\ker\varphi\subseteq\gen{N^{-1}+\Lambda_\tau}$. Como claramente $N^{-1}+\Lambda_\tau\in\ker\varphi$, tenemos la otra contención y podemos concluir que $\ker\varphi=\gen{N^{-1}+\Lambda_\tau}$. Todo esto, junto con el lema \ref{lema:espacio_moduli_Y_0(N)}, nos produce la identificación de los tres conjuntos:
\[
	\begin{tikzcd}[row sep=-1mm,column sep=20mm]
		\mathrm{Isog}_N(\CC) \arrow[r, leftrightarrow,"\Phi_{N,\CC}"] & S_0(N) \arrow[r,leftrightarrow,"\Psi_N"] & Y_0(N) \\
		\big[E_{\tau}\morf{\cdot N} E_{N\tau}\big] \arrow[r,leftrightarrow] & \big[E_\tau,\gen{N^{-1}+\Lambda_\tau}\big] \arrow[r,leftrightarrow] & \tau\Gamma_0(N)
	\end{tikzcd}.
\]
\end{nota}


	Ahora, para $K\subseteq\CC$, consideremos la función
	\[
		\Delta_{N,K}:S_0(N)(K)\lra\AAA^2_K \quad\text{definido por}\quad [E,C]\mapsto \big(j(E),j(E/C)\big),
	\]
donde $E/C$ es la única curva elíptica asociada a una isogenia de $E\ra E/C$ con núcleo $C$ (véase el teorema \ref{thm:kernel-isogenias}). Observe que $\Delta_{N,K}$ está bien definida porque si $[E,C]=[E',C']$, mediante un isomorfismo $f:E\ra E'$, entonces $j(E)=j(E')$.\footnote{\label{footnote-j-inv}Más precisamente, $E$ y $E'$ tienen ecuaciones de Weierstrass y si $E\cong E'$, entonces sus ecuaciones de Weierstrass difieren en un cambio de variable
\[
	X=u^2X'+r,\quad Y=u^3Y'+su^2X'+t \qquad u\in K^*,\; r,s,t\in K,
\]
que no altera el $j-$invariante de las ecuaciones, es decir $j(E)=j(E')$. Equivalentemente tenemos que $j(E)\neq j(E')$ implica $E\not\cong E'$.} Además la composición $E\morf{f}E'\lra E'/C'$ es una isogenia con núcleo $C$ porque $f(C)=C'=\ker(E'\ra E'/C)$. Por la unicidad de $E/C$, necesariamente tenemos que $E/C\cong E'/C'$ y por lo tanto $j(E/C)=j(E'/C')$. Esto prueba que $[E,C]\mapsto(j(E),j(E/C))$ está bien definido.

También podemos probar que $\Delta_{N,K}$ es inyectiva. Supongamos que $[E,C],[E',C']\in S_0(N)(K)$ tales que
\[
	\Delta_{N,K}[E,C]=(j(E),j(E/C))=(j(E'),j(E'/C'))=\Delta_{N,K}[E',C'],
\]
es decir, que $j(E)\neq j(E')$ o $j(E/C)\neq j(E'/C')$. Para esto supongamos que $j(E)=j(E')$. Como $K\subseteq\CC$, $E$ y $E'$ definen curvas sobre $\CC$ que son isomorfas sobre $\CC$, i.e. hay un isomorfismo $g:E(\CC)\ra E'(\CC)$ definido sobre $\CC$. Ahora, la isogenia $E(\CC)\morf{f} E'(\CC)\lra E'(\CC)/C'$ tiene núcleo $C$. Por la unicidad de la curva elíptica $E(\CC)/C$, tenemos que $E(\CC)/C\cong E'(\CC)/C'$ sobre $\CC$ o equivalentemente $j(E)$

Por contradicción, supongamos que $[E,C]=[E',C']$. Esto significa que $E\cong E'$ y así $j(E)=j(E')$ (véase el pie de página ${}^{\ref{footnote-j-inv}}$)



 y por lo tanto $[E,C]\neq[E',C']$. Ahora supongamos que $j(E)=j(E')$, observe que en este caso $j(E/C)\neq j(E/C)$. Si consideramos $E$ y $E'$ sobre $\CC$, entonces son isomorfas sobre $\CC$; sea $f:E\ra E'$ un isomorfismo sobre $\CC$. Si $f(C)=C'$, entonces el núcleo de la isogenia $E\morf{f} E'\lra E'/C'$ es $C$ y tendríamos como antes: $E/C\cong E'/C'$ lo cual contradice $j(E/C)\neq j(E/C)$. Por lo tanto $f(C)\neq C'$ para todo isomorfismo $f:E\ra E'$ sobre $\CC$, en particular $[E,C]\neq[E',C']$ porque de lo contrario habría un isomorfismo $g:E\ra E'$ sobre $K$ que induce un isomorfismo $\CC$ que satisface 


Con esto en mente consideremos la función:
\[
	\Delta_{N,K}: \mathrm{Isog}_N(K)\lra \mathbb{A}^2(K)=K\times K\quad\text{definido por}\quad [E\ra E']\mapsto(j(E),j(E')).
\]
La función está bien definida porque si dos isogenias $E_1\ra E'_1$ y $E_2\ra E'_2$ son isomorfas sobre $K$, entonces hay isomorfismos $E_1\cong E_2$ y $E'_1\cong E'_2$ definidos sobre $K$ lo cual implica que $j(E_1)=j(E_2)\in K$ y $j(E'_1)=j(E'_2)\in K$. También es fácil ver que $\Delta_{N,K}$ es inyectivo ya que si $j(E_1)\neq j(E_2)$ entonces las curvas $E_1$ y $E_2$ no pueden ser isomorfas sobre $K$. Por lo tanto si $(j(E_1),j(E'_1))\neq(j(E_2),j(E'_2))$ entonces $[E_1\ra E'_1]\neq[E_2\ra E'_2]$ y así $\Delta_{N,K}$ es inyectivo. Lo que nos falta hacer es calcular la imagen de $\Delta_{N,K}$.

Observe que cuando $K=\CC$ tenemos que $\Delta_{N,\CC}$ actúa como $[E_\tau\ra E_{N\tau}]\mapsto (j(E_\tau),j(E_{N\tau}))$. En este caso tenemos que $j(E_{N\tau})=j(N\tau)=j_N(\tau)$ y por lo tanto $[E_\tau\ra E_{N\tau}]\mapsto (j(\tau),j_N(\tau))$. Recuerde que $j$ y $j_N$ satisfacen la ecuación modular, i.e. $F(j,j_N)=0$ para algún polinomio $F(X,Y)\in\QQ[X,Y]$. Más precisamente, como $\QQ(j,j_N)$ es de grado de trascendencia 1, $j_N$ es algebraico sobre $\QQ(j)$ y $F$ es el polinomio mínimo de $j_N$ sobre $\QQ(j)$. Por lo tanto la imagen de $\Delta_{N,\CC}$ está contenida en los ceros del polinomio $F(X,Y)$, i.e.
\[
	\Delta_{N,\CC}[E_\tau\ra E_{N\tau}]=(j(\tau),j_N(\tau))\in \{(X,Y)\in\CC\times\CC\mid F(X,Y)=0\}=\Cc(\CC),
\]
donde $\Cc$ es la curva sobre $\QQ$ definida como los ceros de $F(X,Y)$. Es decir que la imagen de $\Delta_{N,\CC}$ está contenida en $\Cc(\CC)$. Ahora este argumento implica el caso $K\subset\CC$ arbitrario, en efecto, si $[E\ra E']\in\mathrm{Isog}_N(K)$, entonces $j(E),j(E')\in K$ y por lo tanto $\Delta_{N,K}[E\ra E']\in\Cc(K)\subseteq\Cc(\CC)$.

Ahora, cuando $K=\QQ$ podemos usar el modelo racional de la curva modular $X_0(N)$ (cf. el teorema \ref{thm:modelo-racional}), que denotamos por $(X^\QQ_0(N),\varphi_N)$, donde $X^\QQ_0(N)$ es una curva proyectiva suave sobre $\QQ$ y $\varphi_N:X_0(N)\ra X^\QQ_0(N)(\CC)$ es un isomorfismo sobre $\CC$ tal que que el isomorfismo inducido en los campos de funciones $\varphi_N^*:\CC(X^\QQ_0(N))\ra\CC(X_0(N))=\CC(j,j_N)$ se restringe a un isomorfismo $\varphi_N^*:\QQ(X^\QQ_0(N))\ra\QQ(j,j_N)$. Los puntos afines de $X^\QQ_0(N)$ son los ceros del polinomio mínimo de $j_N$ sobre $\QQ(j)$, visto como polinomio en $\QQ[X,Y]$, i.e. $X^\QQ_0(N)(K)=\Cc(K)$.

Además, $Y_0(N)$ también tiene un modelo racional $(Y^\QQ_0(N),\varphi_N)$ y se obtiene del modelo de $X_0(N)$ quitándole una cantidad finita de puntos que corresponden a las cúspides de $X_0(N)$. Más precisamente, si $\pi_N:X_0(N)\ra X_0(1)$ es la proyección natural inducida por la función $\tau\Gamma_0(N)\mapsto\tau\SL_2(\ZZ)$, obtenemos un morfismo $\pi_{N,\QQ}$ entre los modelos $X^\QQ_0(N)$ y $X^\QQ_0(1)$ al completar el siguiente diagrama:
	\[
		\begin{tikzcd}
			X_0(N) \arrow[d,"\pi_N"] \arrow[r,"\varphi_N"] & X^\QQ_0(N)(\CC) \arrow[d,dashed,"\pi_{N,\QQ}"] \\
			X_0(1) \arrow[r,"\varphi_1"] & X^\QQ_0(1)(\CC)
		\end{tikzcd}.
	\]
El $j-$invariante induce un isomorfismo $j:X_0(1)\ra\PP^1(\CC)$ donde la cúspide $\infty\SL_2(\ZZ)\in X_0(1)$ es el único punto tal que $j(\infty\SL_2(\ZZ))=\infty=[1,0]\in\PP^1(\QQ)$. Bajo la proyección natural $\pi_N:X_0(N)\ra X_0(1)$, las cúspides de $X_0(N)$ corresponden a la única cúspide de $X_0(1)$. Estos dos hechos implican que la imagen inversa de $\infty$ bajo $j\circ\pi_N$ es el conjunto de cúspides de $X_0(N)$, i.e.
	\begin{gather*}
		\pi_N^{-1}(\infty\SL_2(\ZZ))=
		\pi_N^{-1}(j^{-1}(\infty))=
		\{\tau\Gamma_0(N)\in X_0(N)\mid \tau\SL_2(\ZZ)=\infty\SL_2(\ZZ)\}\\
		\therefore\;\;\pi_N^{-1}(\infty\SL_2(\ZZ))= \{\text{cúspides de}\; X_0(N)\}
	\end{gather*}
Por lo tanto definimos las cúspides de $X^\QQ_0(N)$ como $\varphi_N(\pi_N^{-1}(\infty\SL_2(\ZZ)))$ y definimos $Y^\QQ_0(N)$ como la variedad cuasialgebraica $X^\QQ_0(N)-\varphi_N(\pi_N^{-1}(\infty\SL_2(\ZZ)))$. En particular los puntos afines de la curva $Y^\QQ_0(N)$ satisfacen el mismo polinomio que los puntos afines de $X^\QQ_0(N)$.

Ahora los puntos racionales de $Y^\QQ_0(N)$ son los puntos racionales $X^\QQ_0(N)(\QQ)$ menos una cantidad finita de puntos racionales correspondientes a las cúspides de $X_0(N)$ (cf. la nota después del teorema \ref{thm:modelo-racional}).\marginpar{\scriptsize{agregué una nota en la sección del modela racional de $X_0(N)$}} Afirmamos que la imagen de $\Delta_{N,\QQ} $ es $Y^\QQ_0(N)(\QQ)$. Esto lo enunciamos y probamos de manera más precisa como un corolario del lema \ref{lema:espacio_moduli_Y_0(N)}:
 
\begin{cor}\label{cor:clases-curvas-elipticas-puntos-racionales}
	La función $\Theta:S_0(N)(\QQ)\ra Y^\QQ_0(N)(\QQ)$ definida por $[E,C]\mapsto(j(E),j(E/C))$ es una biyección, es decir las clases de isomorfismo de curvas elípticas con subgrupo cíclico fijo de orden $N$ están en biyección con los puntos racionales de $X_0(N)$ no cuspidales.
\end{cor}

\begin{proof}

Primero descomponemos a $\Theta$ como la composición del inverso de $\Phi_{N,K}$ (cf. \eqref{def:phi-N-K}) con $\Delta_{N,\QQ}$, i.e.
\[
	\begin{tikzcd}[row sep=1mm,column sep=20mm]
		\Theta\; S_0(N)(\QQ) \arrow[r,"\Phi^{-1}_{N,\QQ}"] & \mathrm{Isog}_N(\QQ) \arrow[r,"\Delta_{N,\QQ}"] & X^\QQ_0(N)(\QQ) \\
		\text{[}E,C\text{]} \arrow[r,mapsto] & \text{[}E,E/C\text{]} \arrow[r,mapsto] & (j(E),j(E/C))
	\end{tikzcd}.
\]
Solamente nos falta probar que la imagen de $\Delta_{N,\QQ}$ es $Y^\QQ_0(N)(\QQ)$.

Supongamos que $[E\ra E']\in\mathrm{Isog}_N(\QQ)$ es tal que $\Delta_{N,\QQ}[E\ra E']=(j(E),j(E'))\in\varphi_N(\pi_N^{-1}(\infty\SL_2(\ZZ)))$. Entonces
\[
	\varphi_N^{-1}(j(E),j(E'))\in\pi_N^{-1}(\infty\SL_2(\ZZ))
\]
\end{proof}


\begin{comment}
En esta secci\'on definimos la curva $X_0(N)$ y vemos que parametriza ciertas clases de
isomorfismo de curvas el\'ipticas. Fijamos $N>1$.

Sea $E$ una curva el\'iptica sobre el campo $\QQ(x)$ tal que $j(E)=x$. Sea $P\in E$ un punto de
orden $n$ y sea $C=\{O,P,2P,\ldots,(N-1)P\}$ el subgrupo de $E$ generado por $P$. Toma
$K\subset\overline{\QQ(x)}$ como el campo fijo del subgrupo
$H=\{\sigma\in G_{\QQ(x)}\mid \sigma(C)=C\}$.

Como $(G_{\QQ(x)}:H)<\infty$ (porque $C$ es finito), entonces $K$ es una extensi\'on finita
de $\QQ(x)$. En particular es una extensi\'on de $\QQ$ finitamente generada. Ahora, si
$\overline{\QQ}\cap K=\QQ$ (estamos identificando a $\overline{\QQ}$ con su inclusi\'on en
$\overline{K}$) entonces $K$ es una extensi\'on de $\QQ$ finitamente generada de grado de
trascendencia 1. De esta manera, como la categor\'ia de curvas proyectivas suaves definidas
sobre $\QQ$ (con morfismos dominantes) y la categor\'ia de extensiones de $\QQ$ finitamente
generadas de grado de trascendencia 1 (cf. \cite[\S1.6, corolario 6.12]{HartshorneAG}), podemos
asociar a $K$ una curva proyectiva suave definida sobre $\QQ$ que llamamos $X_0(N)$.

Hay que probar que la elecci\'on de $X_0(N)$ est\'a bien definida, es decir que no depende de
$E$ ni de el subgrupo $C\subset E$ y adem\'as que efectivamente $\overline{\QQ}\cap K=\QQ$ para que
$K$ realmente sea un campo de funciones de una curva. Estas tres proposiciones se siguen del
siguiente teorema:

Sea $E$ una curva el\'iptica sobre $\QQ$ y definimos a $\QQ(E[N])$ como la extensi\'on de
Galois generada por las coordenadas afines de los puntos de $E[N]$. La acci\'on natural
$G_{\QQ(E[N])}\curvearrowright E[N]$ induce una representaci\'on
$\rho:G_{\QQ(E[N])}\ra\GL_{2}(\ZZ/N\ZZ)$ (gracias a la estructura de $E[N]$ dada en la proposici\'on
\ref{prop:estructura_EN}).

\begin{thm}\label{thm:iso_gruposgalois}
  Sea $E$ una curva el\'iptica definida sobre $k=\QQ(x)$ tal que $j(E)=x$. Con la notaci\'on del
  p\'arrafo anterior, la representaci\'on $\rho$ es un isomorfismo, es decir:
  \[
    G_{\QQ(x,E[N])}\cong\GL_{2}(\ZZ/N\ZZ).
  \]
  Adem\'as, $\overline{\QQ}\cap\QQ(x,E[N])=\QQ(\mu_N)$ donde $\mu_N\subset\CC$
  es el conjunto de las $N$-\'esimas raices de la unidad.
\end{thm}

\begin{nota}
  Este resultado es una versi\'on d\'ebil del caso $k=\CC(x)$ donde el isomorfismo es
  $\Gal(\QQ(x,E[N])\mid\QQ(x))\cong\SL_{2}(\ZZ/N\ZZ)$ (cf.
  \cite[cap\'itulo III, \S1, teorema 1 y su corolario]{CornellMFAFLT})
\end{nota}

Ahora explicamos porque la elecci\'on $X_0(N)$ est\'a bien definida:

\begin{cor}
  La curva el\'iptica $X_0(N)$ sobre $\QQ$ existe y no depende de $E$ ni del subgrupo $C$.
\end{cor}

\begin{proof}
  Como mencionamos antes, basta robar que $\overline{\QQ}\cap K=\QQ$ para que $K$ efectivamente
  sea una extensi\'on finitamente generada sobre $\QQ$ de grado de trascendencia 1. Sea $P\in E$
  el generador de $C$. Observa que $\{P\}\subset E[N]$ se puede extender a una base ordenada de
  tal manera que el isomorfismo $G_{\QQ(x,E[N])}\cong\GL_2(\ZZ/N\ZZ)$ del teorema
  \ref{thm:iso_gruposgalois} hace que $H':=\{\sigma\in G_{\QQ(x,E[N])}\mid \sigma(C)=C\}$ sea
  isomorfo a las matrices triangulares inferiores, i.e.
  \[
    H\cong\left\{ \begin{pmatrix}a&0\\b&d\end{pmatrix}:a,d\in(\ZZ/N\ZZ)^{*},\; b\in\ZZ/N\ZZ\right\}.
  \]
  
  Ahora, la funci\'on determinante $\det:\GL_2(\ZZ/N\ZZ)\ra(\ZZ/N\ZZ)^*$ restringida a $H$ sigue
  siendo sobre. Por lo tanto $\QQ(\mu_N)\cap K=\QQ$....
  Si sustituimos la igualdad de la segunda parte del teorema \ref{thm:iso_gruposgalois} en esta
  f\'ormula obtenemos:
  \[
    \QQ=
    \Big(\overline{\QQ}\cap\QQ(x,E[N])\Big)\cap K=
    \QQ(x,E[N])\cap \big(\overline{\QQ}\cap K\big)=\overline{\QQ}\cap K
  \]
  ya que $\overline{\QQ}\cap K\subset\QQ(x,E[N])$.

  Ahora probamos que $X_0(N)$ es independiente de la elecci\'on de $C$. Cambiar de subgrupo $C$ es
  cambiar de punto $P$ de orden $N$. Sean $P'\in E$ otro punto de orden $N$, $C'\subset E[N]$ el
  subgrupo c\'iclico generado por $P'$ y $H'$ el subgrupo de $G_{\QQ(x,E[N])}$ de fija a $C'$. De la
  misma manera extendemos $\{P'\}$ a otra base de $E[N]$. Este cambio de base modifica el
  isomorfismo $G_{\QQ(x,E[N])}\cong\GL_2(\ZZ/N\ZZ)$ mediante una conjugaci\'on por la matriz de cambio
  de base. En particular la imagen de $H'$ en $\GL_2(\ZZ/N\ZZ)$ es un conjugado de la imagen de $H$.
  Por lo tanto existe un $\sigma\in G_{\QQ(x,E[N])}$ tal que $H'=\sigma H\sigma^{-1}$. Por lo tanto
  el campo fijo $K'$ de $H'$ es simplemente $\sigma(K)$, es decir $K\cong K'$. Gracias a la
  equivalencia de categor\'ias mencionada al principio de la secci\'on, $X_0(N)$ es isomorfo a
  cualquier curva proyectiva suave con campo de funciones $K'$ y por lo tanto $X_0(N)$ es
  independiente de la elecci\'on de $C$.

  Por \'ultimo probamos que $X_0(N)$ es independiente de la elecci\'on de la curva $E/\QQ(x)$....  
\end{proof}

Como consecuencia de este corolario, cada curva proyectiva $X_0(N)$ sobre $\QQ$ tiene asociado una
curva el\'iptica $E/\QQ(x)$ (con $j(E)=x$) y un subgrupo c\'iclico $C\subset E$ de orden $N$ tal que
el campo de funciones $K$ de $X_0(N)$ es el campo fijo de $H=\{\sigma\in G_{\QQ(x)}\mid \sigma(C)=C\}$.
La inclusi\'on $\QQ(x)\hookrightarrow K$ induce un morfismo de curvas $X_0(N)\ra\PP^1(\QQ)$. 
A un punto en la imagen inversa de $\infty\in\PP^1(\QQ)$ se le llama una \emph{c\'uspide} de
$X_0(N)$.

Tambi\'en podemos considerar a $X_0(N)$ como una curva proyectiva sobre $\CC$; en este caso su
campo de funciones es $K\otimes_{\QQ}\CC$. Como en el p\'arrafo anterior, la inclusi\'on
$\CC(x)\hookrightarrow K\otimes\CC$  determina un morfismo $X_0(N)(\CC)\ra\PP^1(\CC)$. Sea
$S\subseteq\PP^1(\CC)$ un subconjunto y $S^{\text{c}}$ su complemento en $\PP^1(\CC)$. Denotamos
$X_0(N)(\CC)_S$ como la imagen inversa de $S^{\text{c}}$ bajo $X_0(N)(\CC)\ra\PP^1(\CC)$.

Estamos en posici\'on de estudiar c\'omo parametriza $X_0(N)$ a algunas curvas el\'ipticas, pero
primero debemos definir una categor\'ia nueva. Los objetos son parejas $(E,C)$ donde $E/\CC$ es
una curva el\'iptica y $C\subset E$ es un subgrupo c\'iclico de orden $N$. Los morfismos
$(E,C)\ra(E',C')$ son isomorfismos de curvas $\varphi:E\ra E'$ tales que $\varphi(C)=C'$. A la
clase de isomofismo de $(E,C)$ la denotamos por $[E,C]$ y al conjunto de clases de isomorfismo
lo denotamos por $\text{El}_0(N)(\CC)$. Adem\'as, si $S\subseteq\PP^1(\CC)$ entonces escribimos
\[
  \text{El}_0(N)(\CC)_S:=\{[E,C]\in\mathrm{El}_0(N)(\CC)\mid j(E)\not\in S\}.
\]
Similarmente denotamos por $\mathrm{Toro}_0(N)$ al conjunto de clases de isomorfismo de parejas
$(T,C)$ donde $T$ es un toro complejo de dimensi\'on 1 (i.e. $T\cong\CC/\Lambda$ para alguna ret\'icula)
y $C\subset T$ es un subgrupo c\'iclico de orden $N$.


Ahora, sea $x\in X_0(N)(\CC)$. Como $X_0(N)(\CC)$ es una curva suave, $x$ determina un anillo de
valoraci\'on discreta $\Oo_x\subset K\otimes\CC$ con ideal maximal $\m_x$. Si $E$ tiene buena
reducci\'on en $\m_x$, entonces la reducci\'on m\'odulo $\m_x$ produce una curva el\'iptica
$E_x/\CC$. La restricci\'on de la reducci\'on m\'odulo $\m_x$ a $E[n]\ra E_x[N]$ es inyectiva y
as\'i la reducci\'on m\'odulo $\m_x$ del punto $P\in E[N]$ es un punto $P_x\in E_x[N]$ de orden
$N$ que genera un subgrupo c\'iclico $C_x\subset E_x$ de orden $N$.

Con estas consideraciones podemos enunciar el resultado m\'as importante de esta secci\'on:
\begin{thm}
  Sean $E/\QQ(x)$ una curva el\'iptica tal que $j(E)=x$, $S\subseteq\PP^1(\CC)$ un subconjunto
  que contiene a todos los lugares donde $E$ tiene mala reducci\'on, $\{Q,P\}$ una $\ZZ/N\ZZ$-base
  de $E[N]$ y $C\subset E$ el subgrupo c\'iclico generado por $P$, entonces tenemos el siguiente
  diagrama conmutativo de funciones biyectivas:
  \[
    \begin{tikzcd}[column sep=large, row sep=large]
      X_0(N)(\CC)_S \arrow[r,"(i)"] \arrow[d,"(ii)"'] &
      \mathrm{El}_0(N)(\CC)_S \arrow[d,"(iv)"] \\
      \HH/\Gamma_0(N) \arrow[r,"(iii)"'] & \mathrm{Toro}_0(N)
    \end{tikzcd}
  \]
  donde las funciones est\'an dadas por:
  \begin{enumerate}[label=\emph{\roman*})]
  \item $x\mapsto [E_x,C_x]$.
  \item La restricci\'on del isomorfismo $X_0(N)(\CC)\cong\HH^*/\Gamma_0(N)$ de superficies de
    Riemann.
  \item $[z]\mapsto [\CC/\Lambda_z,\gen{\tfrac{1}{N}+\Lambda_z}]$ donde $\Lambda_z:=z\ZZ\oplus\ZZ$
    es una ret\'icula de $\CC$.
  \item $[E,C]\mapsto [E(\CC),C]$.
  \end{enumerate}
\end{thm}

\begin{proof}
  La prueba de que ($i$) es biyectiva se sigue de
  \cite[cap\'itulo III, \S1.3, proposici\'on 1]{CornellMFAFLT}, la biyectividad de ($ii$) se sigue
  de \cite[cap\'itulo III, \S1.10, proposici\'on 6]{CornellMFAFLT}, la biyectividad de ($iii$) se
  sigue de \cite[cap\'itulo III, \S1.10, proposici\'on 7]{CornellMFAFLT} y la biyectividad de
  ($iv$) se sigue de \cite[cap\'itulo III, \S1.8, proposici\'on 5]{CornellMFAFLT}.
\end{proof}

\begin{defin}
  Una curva el\'iptica $E/\QQ$ es \emph{modular} si existe una funci\'on holomorfa no constante
  $X_0(N)\ra E$ para alguna $N$.
\end{defin}
\end{comment}

\end{document}
