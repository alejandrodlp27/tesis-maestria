\documentclass[../../tesis_maestria]{subfiles}
\begin{document}
%\section{}

En este cap\'itulo formulamos y probamos los resultados necesarios para ver como la conjetura
de Shimura-Taniyama-Weil semiestable se sigue de la conjetura del levantamiento modular
semiestable:

\begin{thm}\label{thm:smlc}(CLMS)
  Sea $p$ un primo impar y $E$ una curva el\'iptica semiestable definida sobre $\QQ$ tal que
  cumple las siguientes dos propiedades:
  \begin{enumerate}[label=\emph{\roman*})]
  \item $\rhop$ es irreducible
  \item\label{cond_ii} Existe una eigenforma $f\in S_2(\Gamma_0(N))$ y un ideal primo
    $\fP\subset\Oo_f$ tal que, para casi todo n\'umero primo $q$, se tiene
    \[
      a_q(f) \equiv q+1-\# E(\FF_q) \Mod{\fP}.
    \]
    Entonces $E$ es modular.
  \end{enumerate}
\end{thm}

A la proposici\'on l\'ogica que postula el teorema \ref{thm:smlc}, aplicado a un primo impar $p$,
la llamaremos CLMS$(p)$.

En las siguientes tres secciones vamos a probar tres resultados fundamentales que reducen la
prueba de STW-semiestable a la prueba de CLMS(3) y CLMS(5). En la cuarta secci\'on usamos estos
resultados para probar que efectivamente CLMS(3) y CLMS(5) implican STW-semiestable.

\end{document}
