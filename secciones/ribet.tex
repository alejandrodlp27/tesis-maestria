\section{Las conjeturas de Serre y un teorema de Ribet}

En esta sección describimos el teorema \ref{thm:ribet} y cómo surge de unas conjeturas que hizo Serre sobre las formas modulares asociadas a ciertas representaciones de Galois. En \cite{Serre87} y en unas cartas a J. P. Mestre en \cite{lettreMestre}, Serre formula varias conjecturas precisas sobre las formas modulares asociadas a representaciones de Galois sobre campos finitos. 


Ahora que tenemos algunas propiedades de la curva de Frey, ahora estudiamos las representaciones de Galois asociadas a $E$ y sus características de reducción.

\begin{prop}
Sea $E=E_{a,b,c,p}$ una curva de Frey asociada a un contraejemplo de UTF($p$). Denotamos $\bar{\rho}_{a,b,c,p}=\rhop:\GQ\ra\GL_2(\FF_p)$ por la representación de Galois de los puntos de $p-$torsión de $E$ (cf. definición \ref{def:rep-galois-N-torsion}). Entonces $\bar{\rho}_{a,b,c,p}$ satisface:
\begin{enumerate}[label=(\roman*)]
	\item $\bar{\rho}_{a,b,c,p}$ es absolutamente irreducible e impar.
	\item $\bar{\rho}_{a,b,c,p}$ es no-ramificado fuera de $2p$.
	\item $\bar{\rho}_{a,b,c,p}$ es plano en $p$
\end{enumerate}
\end{prop}

\begin{proof} Escribimos $\bar{\rho}:=\bar{\rho}_{a,b,c,p}$.
\begin{enumerate}[label=(\emph{\roman*})]
	\item Por el corolario \ref{cor:det_de_rhop}, el determinante de $\bar{\rho}$ es el caracter ciclotómico $\bar{\chi}_p$ módulo $p$ y por lo tanto $\bar{\rho}$ es impar. Si probamos que es irreducible, por la proposición \ref{prop:irred_equiv_absirred} tendremos que $\bar{\rho}$ es absolutamente irreducible. Para probar la irreducibilidad, supongamos que $\bar{\rho}$ es reducible.
\end{enumerate}
\end{proof}

