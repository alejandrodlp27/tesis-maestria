\section{El teorema de Langlands-Tunnel y la modularidad de $\rhot$}\label{sec:langlands_tunnell}
Sea $E$ una curva el\'iptica sobre $\QQ$ y sea $\rhot$ la representaci\'on asociada a sus puntos de $3-$torsi\'on (c.f. la secci\'on \ref{sec:rep_galois}). En esta secci\'on, probamos c\'omo la modularidad de $\rhot$ se sigue de un teorema celebrado de Langlands \cite{LanglandsBCFGL} y Tunnel \cite{TunnellACFROOT}. La versi\'on de su teorema que vamos a usar es:

\begin{thm}\label{thm:langlandstunnell}(Langlands-Tunnell)
  Sea $\sigma:G_{\QQ}\ra\GL_2(\CC)$ una representaci\'on continua, impar, irreducible y tal que $\sigma(\GQ)/\{\pm1\}\subset\PGL_2(\CC)$ es un subgrupo soluble. Entonces existe una forma primitiva $g\in S_1^{\mathrm{new}}(\Gamma_0(N),\chi)$ (para alg\'un entero $N$ y un caracter $\chi$ m\'odulo $N$) tal que para casi todo primo $q$ se tiene
  \[
    a_q(g)=\tr(\sigma(\Fr_q)).
  \]
\end{thm}

La prueba de este teorema se divide en tres casos: cuando $\sigma(\GQ)$ es isomorfo a $S_4$ (las simetr\'ias del octahedro), $A_4$ (las simetr\'ias del tetrahedro) y $D_{2n}$ (el grupo dih\'edrico). La prueba en el caso dih\'edrico es debido a los trabajos de Hecke y Maass. El caso tetrah\'edrico es debido a Langlands y el caso octah\'edrico lo empez\'o Langlands en \cite{LanglandsBCFGL} y lo termin\'o Tunnell en \cite{TunnellACFROOT}.

%En realidad el teorema \ref{thm:langlandstunnell} viene formulado (y probado) con el lenguaje %de formas automorfas cuspidales de peso 1:
%
%\begin{thm}
%  Para cada representaci\'on $\sigma:G_{\QQ}\ra\GL_2(\CC)$ continua, irreducible, impar y con imagen soluble, hay una representaci\'on automorfa, cuspidal, de peso 1 sobre $\GL_2(\AAA_{\QQ})$, denotada por $\pi(\sigma)$ tal que para casi todo primo $q$ se tiene que
%  \[
%    \tr(t_{\pi_q})=\tr(\sigma(\Fr_q)),
%  \]
%donde $\AAA_{\QQ}$ es el anillo de adeles racional y $t_{\pi_q}$ es la \emph{clase de Langlands} asociado al componente no-ramificado $\pi_q$ de $\pi(\sigma)$.
%\end{thm}
%
%No explicamos las definiciones necesarias para este teorema porque se salen del rango de este documento. Simplemente mencionamos el teorema y referimos al lector a \cite[cap\'itulo 6 de Stephen Gelbart]{CornellMFAFLT} para el enunciado y la prueba.

El teorema de Langlands-Tunnell es un caso particular de la conjetura de reciprocidad de Langlands porque establece una correspondencia biyectiva entre formas primitivas de peso 1 y representaciones irreducibles automorfas de peso 1 sobre $\GL_2(\AAA_{\QQ})$ (v\'ease la secci\'on \S2.5  del cap\'itulo de Stephen Gelbart de \cite{GelbartTL} para m\'as detalles).

El prop\'osito de esta secci\'on es probar el siguiente teorema:

\begin{thm}\label{thm:modularidad_rhot}
  Sea $E$ una curva el\'iptica sobre $\QQ$. Si $\rhot:G_{\QQ}\ra \GL_2(\FF_3)$ es irreducible, entonces $\rhot$ es modular.
\end{thm}

La prueba de este teorema se divide en cuatro pasos que en seguida desciribimos a grandes rasgos:
\begin{enumerate}
\item Levantamos la representaci\'on $\rhot$ a una representaci\'on $\GQ\ra\GL_2(\CC)$ que sea impar, con imagen en $\PGL_2(\CC)$ soluble e irreducible;
\item Aplicamos el teorema de Langlands-Tunnell para obtener una forma primitiva de peso 1 asociada al levantamiento de $\rhot$;
\item Multiplicamos la forma primitiva por una serie de Eisenstein de peso 1 para obtener una forma cuspidal de peso 2 que, aunque no es una eigenforma, s\'i es una eigenforma m\'odulo alg\'un ideal del campo num\'erico de la forma primitiva del paso anterior y que contiene a $(3)\subset\ZZ$;  
\item Aplicamos el lema de levantamiento de Deligne-Serre (c.f. lema \ref{lema:deligne_serre}) para obtener una genuina eigenforma asociada a $\rhot$ y as\'i concluir que $\rhot$ es modular.
\end{enumerate}

\begin{proof}
  El primer paso de la demostraci\'on es levantar la representaci\'on $\rhot$ a una representaci\'on $\rho:G_{\QQ}\ra \GL_2(\CC)$ que sea irreducible, impar y soluble.

  Primero observamos que el ideal primo $(1+\sqrt{-2})\subset\ZZ[\sqrt{-2}]$ contiene al ideal primo $(3)\subset\ZZ$ cuya factorizaci\'on en $\QQ(\sqrt{-2})$ es $(3)=(1+\sqrt{-2})(1-\sqrt{-2})$. Con la \emph{identidad fundamental}\footnote{La identidad fundamental es una relaci\'on num\'erica entre la factorizaci\'on de ideales en una extensi\'on finita de dominios de Dedekind con el grado de la extensi\'on de sus campos de cocientes. M\'as precisamente, fijamos $\Oo$ un dominio de Dedekind con campo de cocientes $K$ y sea $L$ una extensi\'on separable de $K$ de grado $n$ con $\Oo'$ la cerradura integral de $\Oo$ en $L$. Sea $\p\subset\Oo$ un ideal primo cuya extensi\'on en $\Oo'$ se factoriza en potencias de ideales primos como $\p\Oo'=\fP_1^{e_1}\cdots\fP_r^{e_r}$. Adem\'as escribimos $f_i:=[\Oo'/\fP_i:\Oo/\p]$ como el grado inercial de $\fP_i$ sobre $\p$.

    La identidad fundamental dice que $e_1f_1+\cdots+e_r f_r=n$. Si adem\'as la extensi\'on $L/K$ es de Galois (como el caso $\QQ(\sqrt{-2})/\QQ$ del texto) tenemos que $e:=e_1=\cdots=e_r$ y $f:=f_1=\cdots=f_r$ y la identidad fundamental se reduce a $n=efr$. El caso general es la proposici\'on 8.2 de la secci\'on 1.8 de \cite{NeukirchANT}, el caso cuando la extensi\'on es de Galois viene en $\S1.9$.} para la factorizaci\'on de ideales primos en extensiones de campos deducimos inmediatamente que el grado inercial de $(1+\sqrt{-2})$ sobre $(3)$ es
  \[
    \left[\tfrac{\ZZ[\sqrt{-2}]}{(1+\sqrt{-2})}:\tfrac{\ZZ}{3\ZZ}\right]=1
  \]
y por lo tanto
  \[
    \frac{\ZZ[\sqrt{-2}]}{(1+\sqrt{-2})}\cong \FF_3.
  \]

  Con esta expresi\'on para $\FF_3$ queremos definir un homomorfismo inyectivo $\Psi:\GL_2(\FF_3)\ra\GL_2(\ZZ[\sqrt{-2}])$. Para esto tomamos
  \[
    A=\begin{pmatrix}-1&1\\-1&0\end{pmatrix}\quad \text{y} \quad B=\mat{1}{-1}{1}{1}.
  \]
  como unos generadores de $\GL_2(\FF_3)$. Se puede verificar directamente las siguientes relaciones:\marginpar{\scriptsize checar que esa es la \'unica relaci\'on.}
  \[
    A^3=\Id\quad\text{y}\quad B^8=\Id.
  \]
  Además, estos exponentes son los enteros mínimos positivos que satisfacen estas relaciones.
  
  Como $A$ tiene orden tres y $B$ tiene orden ocho, entonces la intersecci\'on $\gen{A}\cap\gen{B}=\{\Id\}$. Por lo tanto $\gen{A}\gen{B}=\{A^nB^m\mid 1\leq n\leq 3,1\leq m\leq 8\}$ tiene 24 elementos y as\'i $\gen{A,B}$ tiene al menos 24 elementos. Como $\gen{A,B}$ no es abeliano, tiene m\'as de 24 elementos (e.g. $BA\in\gen{A,B}-\gen{A}\gen{B}$) y as\'i, por ser subgrupo de $\GL_2(\FF_3)$ que tiene 48 elementos (cf. la sección \ref{sec:subgruposdecongruencia}), $\gen{A,B}$ tiene 48 elementos. Por lo tanto que $A$ y $B$ efectivamente generan a $\GL_2(\FF_3)$.
  
Ahora, definimos $\Psi$ sobre los generadores como
  \[
    \Psi(A):= A \quad,\quad
    \Psi(B):=
    \begin{pmatrix}1 & -1 \\ -\sqrt{-2} & -1+\sqrt{-2}\end{pmatrix}.
  \]
Observe que 
\[
	\Psi(B)^4=\mat{1}{-1}{-\sqrt{-2}}{-1+\sqrt{-2}}^4=\mat{1+\sqrt{-2}}{-\sqrt{-2}}{2}{-1-\sqrt{-2}}^2=-\Id.
\]
Esto implica que $\Psi$ preserva las relaciones de los generadores de $\GL_2(\FF_3)$ y as\'i $\Psi$ es un homorfismo de grupos.

  La proyecci\'on natural $\ZZ[\sqrt{-2}]\epi\FF_3$ induce un epimorfismo $\nu:\GL_2(\ZZ[\sqrt{-2}])\epi\GL_2(\FF_3)$. De su definici\'on se puede verificar que $\Psi$ es una secci\'on de $\nu$ y cabe en el siguiente diagrama conmutativo:
  
\begin{equation}\label{cd:seccion_de_nu}
  \begin{tikzcd}
    \GL_2(\FF_3) \arrow[r,"\Psi"] \arrow[dr,"\Id"'] & \GL_2(\ZZ[\sqrt{-2}]) \arrow[d,"\nu"] \\
    & \GL_2(\FF_3)
  \end{tikzcd}
\end{equation}
  
  Gracias a la conmutatividad de este diagrama podemos calcular la traza y el determinante de la representaci\'on $\Psi$. Si $C\in\GL_2(\FF_3)$, tenemos que
  \begin{equation}
    \label{eq:traza_mod_3}
    \tr\big(\Psi(C)\big)\equiv\tr(C) \Mod{1+\sqrt{-2}}.
  \end{equation}
  Para el determinante de $\Psi$ tenemos la congruencia m\'as fuerte
    \begin{equation}
    \label{eq:det_mod_3}
    \det\big(\Psi(C)\big)\equiv\det(C) \Mod{3},
  \end{equation}
  que se verifica sobre los generadores y se extiende a todo $\GL_2(\FF_3)$ por multiplicatividad del determinante.
  
Como $\ZZ[\sqrt{-2}]\subseteq\CC$, consideramos la composici\'on $\GL_2(\FF_3)\morf{\Psi}\GL_2(\ZZ[\sqrt{-2}])\hookrightarrow\GL_2(\CC)$ que tambi\'en denotamos por $\Psi$. Con esta notaci\'on definimos:
  \[
    \rho:=\Psi\circ\rhot:G_{\QQ}\lra\GL_2(\CC).
  \]
  Para poder aplicar el Teorema de Langlands-Tunnell necesitamos probar que $\rho$ cumple cuatro cosas:
  \begin{enumerate}[label=\emph{\roman*})]
  \item $\rho$ es continua.

    \-\;\; Como $\GL_2(\FF_3)$ tiene la topolog\'ia discreta por lo tanto $\Psi$ es autom\'aticamente continua. Como $\rhot$ tambi\'en es continua, concluimos que $\rho$ tambi\'en lo es.
  \item\label{inciso_sigma_impar} $\rho$ es impar.

    \-\;\; Sea $\fc\in G_{\QQ}$ la conjugaci\'on compleja. Claramente $\fc^2=1$ lo cual implica que $\rho(\fc)^2=\Id$ y as\'i $\det(\rho(\fc))$ satisface la ecuaci\'on $x^2-1=0$. Por lo tanto $\det(\rho(\fc))=\pm1$.

    \-\;\; Por otro lado, \eqref{eq:det_mod_3} nos dice que
    \[
      \det\big(\rho(\fc)\big) = \det\big(\Psi\big(\rhot(\fc)\big)\big) \equiv
      \det\big(\rhot(\fc)\big) \Mod{3},
    \]
    pero por el corolario \ref{cor:det_de_rhop} sabemos que $\det\rhot=\bar{\chi}_3$, el caracter ciclot\'omico m\'odulo 3. Como $\bar{\chi}_3$ es el caracter inducido por la acci\'on de $\GQ$ sobre $\QQ(e^{2\pi i/3})$ tenemos que $\bar{\chi}_3(\fc)$ act\'ua como conjugaci\'on compleja y as\'i $\bar{\chi}_3(\fc)=-1\in(\ZZ/3\ZZ)^*$. Por lo tanto
    \[
      \det\big(\rho(\fc)\big) \equiv \bar{\chi}_3(\fc)= -1 \Mod{3}.
    \]
    Como ya ten\'iamos que $\det(\rho(\fc))=\pm 1$, la congruencia anterior implica que $\det(\rho(\fc))=-1$ porque $1\not\equiv -1\Mod{3}$. Por lo tanto $\rho$ es impar.
  \item $\rho$ es soluble.
    
    \-\;\; Primero afirmamos que
    \begin{equation}\label{eq:pgl23_S4}
      \PGL_2(\FF_3):=\frac{\GL_2(\FF_3)}{\{\Id,-\Id\}}\cong S_4
    \end{equation}
    donde $S_4$ es el grupo de permutaciones de un conjunto de cuatro elementos.

    \-\;\; La acción natural $\GL_2(\FF_3)\curvearrowright\PP^1(\FF_3)$ no es fiel pues las matrices escalares actúan trivialmente, es decir el núcleo de esta acción contiene a $\{\Id,-\Id\}$ (aquí, los únicos escalares son 1 y -1).
    
    \-\;\; Ahora probamos que no hay otras matrices en el núcleo. Supongamos que $A=(a_{ij})\in\GL_2(\FF_3)$ fija a todos los elementos $[x,y]\in\PP^1(\FF_3)$. En particular fija a la base $\{(1,0),(0,1)\}$ de $\FF_3\times\FF_3$. De esta manera obtenemos las siguientes fórmulas:
    \begin{align*}
			[1,0]=\mat{a_1}{a_2}{a_3}{a_4}\begin{bmatrix}1\\0\end{bmatrix}=[a_1,a_3],\\
			[0,1]=\mat{a_1}{a_2}{a_3}{a_4}\begin{bmatrix}0\\1\end{bmatrix}=[a_2,a_4].
    \end{align*}
    Éstas implican que $a_3=0=a_2$ y $\abs{a_1}=1=\abs{a_4}$. Supongamos que $a_1$ y $a_4$ tienen signo distinto, i.e. $a_4=-a_1$. Con la fórmula del determinante deducimos que $1=\det A=-a_1^2$, pero esto es imposible porque $-1\in\FF_3$ no es un cuadrado. Por lo tanto $a_1=\pm1=a_4$ y as\'i $A=\pm\Id$.
    
    \-\;\; Hemos probado que el núcleo de la acción $\GL_2(\FF_3)\curvearrowright\PP^1(\FF_3)$ es $\{\pm\Id\}$. Por lo tanto desciende a una acción fiel $\PGL_2(\FF_3)\act\PP^1(\FF_3)$. Equivalentemente, hay una homomorfismo inyectivo $\PGL_2(\FF_3)\inc S_4$ porque $\PP^1(\FF_3)$ tiene 4 elementos: los tres de $\FF_3$ y un punto al infinito. Por otro lado $\GL_2(\FF_3)$ tiene $48$ elementos, entonces $\PGL_2(\FF_3)$ tiene $48/2=24=4!$ elementos. Por lo tanto la inclusión $\PGL_2(\FF_3)\inc S_4$ es en realidad un isomorfismo.

    \-\;\; Una vez establecido \eqref{eq:pgl23_S4}, vamos a ver que la imagen de $\rho$ en $\PGL_2(\CC)$ es soluble. Como $\Psi$ es inyectivo, podemos hacer la identificación $\GL_2(\FF_3)\cong\Psi(\GL_2(\FF_3))\subset\GL_2(\CC)$. Por otro lado tenemos que
    \[
      \Psi(\GL_2(\FF_3))\cap\{\la\Id\}_{\la\in\CC}=\{\pm\Id\}.
    \]
    En efecto, si $\la\Id\in\Psi(\GL_2(\FF_3))$ entonces $\la$ es una ra\'iz de la unidad porque $\Psi(\GL_2(\FF_3))$ es un grupo de orden finito y como $\Psi(\GL_2(\FF_3))\subset\GL_2(\ZZ[\sqrt{-2}])$, esto implica que $\la=\pm1$ porque $\ZZ[\sqrt{-2}]$ no contiene otras raíces de la unidad.

    \-\;\; Por lo tanto tenemos la inclusión
\[
	\PGL_2(\FF_3)\cong\frac{\Psi(\GL_2(\FF_3))}{\{\pm\Id\}}\subset\frac{\GL_2(\CC)}{\{\la\Id\}_{\la\in\CC}}=\PGL_2(\CC).
\]
Como $\rho=\Psi\circ\rhot$, entonces $\rho(\GQ)$ es un subgrupo de $\Psi(\GL_2(\FF_3))$. De esta manera $\rho(\GQ)/\{\pm1\}$ es isomorfa a un subgrupo de $\PGL_2(\FF_3)\cong S_4$, que es un grupo soluble\footnote{En efecto, $\{1\}\triangleleft \FF_2\times\FF_2 \triangleleft A_4 \triangleleft S_4$ es una serie normal cuyos cocientes son abelianos.}. Por lo tanto la imagen de $\rho(\GQ)$ en $\PGL_2(\CC)$ es soluble.

  \item $\rho$ es irreducible.

    \-\;\; Supongamos que $\rho$ es una representaci\'on reducible. Como $G_{\QQ}$ es compacto y $\rho$ es una representaci\'on de dimensi\'on 2, $\rho$ se descompone como suma de representaciones irreducibles de dimensi\'on 1\footnote{Toda representaci\'on de un grupo finito en un espacio vectorial de dimensi\'on finita se descompone como suma directa de representaciones irreducibles. La prueba de este hecho es una aplicaci\'on elemental de inducci\'on sobre la dimensi\'on del espacio vectorial (c.f. \cite[\S1.4]{SerreLROFG}). Hay dos maneras de generalizar este hecho a $\rho$: observar que $\rho$ se factoriza a trav\'es del cociente finito $\GQ/\Gal(K_{\rho}|\QQ)$ (v\'ease la nota anterior al ejemplo \ref{ej:car_ciclo_modN}) o usar la compacidad de $\GQ$ y la existencia de su medida de Haar para generalizar la demostraci\'on a grupos compactos no necesariamente finitos (c.f. \cite[\S4.3]{SerreLROFG}).} Esto implica que $\rho(\GQ)\subseteq\GL_2(\CC)$ es un subgrupo abeliano. En efecto, si escribimos $\rho=\rho_1\oplus\rho_2$, entonces despu\'es de elegir una base adecuada, tenemos
    \[
      \rho(s)=\mat{\rho_1(s)}{0}{0}{\rho_2(s)}\qquad\forall s\in\GQ.
    \]
De aqu\'i es claro ver que $\rho(\GQ)\subset\GL_2(\CC)$ es abeliano. Adem\'as como $\Psi:\GL_2(\FF_3)\ra\GL_2(\CC)$ es inyectivo, tenemos que $\rhot(\GQ)\cong\Psi(\rhot(\GQ))=\rho(\GQ)$. Por lo tanto $\rhot(\GQ)$ es un subgrupo abeliano de $\GL_2(\FF_3)$.

%De esta manera, si $s,s'\in \GQ$, entonces:
%    \begin{align*}
%      \rho(s)\rho(s')
%      &=\mat{\rho_1(s)}{0}{0}{\sigma_2(s)}\mat{\rho_1(s')}{0}{0}{\rho_2(s')}=
%        \mat{\rho_1(s)\rho_1(s')}{0}{0}{\rho_2(s)\rho_2(s')}\\
%      &=\mat{\rho_1(s')\rho_1(s)}{0}{0}{\rho_2(s')\rho_2(s)}=
%        \mat{\rho_1(s')}{0}{0}{\sigma_2(s')}\mat{\rho_1(s)}{0}{0}{\rho_2(s)}\\
%      & =\rho(s')\rho(s).
%      \end{align*}

\-\;\; Ahora sea $S_0:=\rho(s_0)\in\rhot(\GQ)$ arbitrario. Sea $\la$ un valor propio del endomorfismo $S_0:\FF_3\times\FF_3\ra\FF_3\times\FF_3$ que podemos tomar en alguna extensi\'on finita $F$ de $\FF_3$. Si consideramos a $S_0$ como elemento de $\GL_2(F)$ bajo la inclusi\'on $\GL_2(\FF_3)\subset\GL_2(F)$, podemos definir el endomorfismo $S_1:=S_0-\la\Id$ de $F\times F$ y denotamos por $W$ a su n\'ucleo. Como $\la$ es valor propio de $S_0$, entonces $W\neq0$.

\-\;\; Por otro lado, para toda $s\in\GQ$ tenemos que:
\begin{align*}
  \rhot(s)\circ S_1
  &=\rhot(s)(S_0-\la\Id)=\rhot(s)S_0-\rhot(s)\la\Id\\
  &\overset{*}{=}S_0\rhot(s)-\la\rhot(s)=(S_0-\la\Id)\rhot(s)\\
  &=S_1\circ\rhot(s),
\end{align*}
donde el paso (*) se sigue de que $\rhot(\GQ)$ es abeliano y que las matrices escalares conmutan con todas las matrices. Esta igualdad nos permite deducir que para toda $x\in W$:
\[
  S_1\big(\rhot(s)(x)\big)=\rhot(s)\big(S_1(x)\big)=\rhot(s)(0)=0,
\]
lo cual implica que $\rhot(s)(x)\in W$ para toda $s\in\GQ$. Por lo tanto $W\subseteq F\times F$ es un subespacio $\GQ-$estable bajo la representaci\'on $\GQ\morf{\rhot}\GL_2(\FF_3)\hookrightarrow\GL_2(F)$.

\-\;\; Ahora, como $\rhot$ es irreducible por hip\'otesis, la proposici\'on \ref{prop:irred_equiv_absirred} implica que $\rhot$ es absolutamente irreducible. En particular la representaci\'on $\GQ\morf{\rhot}\GL_2(\FF_3)\hookrightarrow\GL_2(F)$ es irreducible. Como el subespacio invariante $W=\ker S_1$ es distinto de 0, necesariamente tenmos que $W=F\times F$, i.e. $S_0-\la\Id=0$ o equivalentemente $\rhot(s_0)=\la\Id$. La elecci\'on de $s_0\in\GQ$ fue arbitraria, entonces podemos tomar $s_0=\fc$ la conjugaci\'on compleja. Esto produce una contradicci\'on porque $\rhot(\fc)$ no puede ser una matriz escalar porque tiene valores propios distintos como hab\'iamos establecido cuando vimos que $\rho$ era impar. La contradicci\'on surge de asumir que $\rho$ era reducible, entonces concluimos que $\rho$ es irreducible.
\end{enumerate}


Despu\'es de probar estas cuatro propiedades, podemos aplicar el Teorema de Langlands-Tunnell a la representaci\'on $\rho$: existe una forma primitiva $g\in S_1^{\mathrm{new}}(\Gamma_0(N),\chi)$ para alguna $N\in\NN$ y alg\'un caracter $\chi:(\ZZ/N\ZZ)^*\ra\CC^*$, con serie de Fourier
  \[
    g(z)=\sum_{n=1}^{\infty}a_n(g)e^{2\pi i nz},
  \]
cuyos coeficientes cumplen que, para casi todo primo $q$,
  \begin{equation}\label{eq:g_tr_fourier}
    a_q(g)=\tr\big(\rho(\Fr_q)\big).
  \end{equation}
  Recuerde que los coeficientes de Fourier de $g$ est\'an contenidos en su campo num\'erico $K_g:=\QQ(\{a_n(g),\chi(n)\}_{n\geq1})$ que es una extensi\'on finita de $\QQ$. Denotamos por $\Oo_g$ al anillo de enteros de $K_g$. De hecho sucede algo m\'as fuerte, los coeficientes de Fourier son enteros de $K_g$, i.e. $a_n(g)\in\Oo_g$ (v\'ease la nota despu\'es de la proposici\'on \ref{prop:camponumerico}). Por lo tanto podemos calcular la traza y el determinante de $\rho$ m\'odulo alg\'un ideal primo de $\Oo_g$ que contenga al ideal $(1+\sqrt{-2})$ (v\'ease la congruencia \eqref{eq:traza_mod_3}).
  
  Sea $\fP\subset\Oo_g$ un ideal primo que contiene al ideal $(1+\sqrt{-2})$. Para casi todo primo $q$ tenemos:
  \begin{align}
    a_q(g)
    &=\tr\big(\rho(\Fr_q)\big)=\tr\big(\Psi(\rhot(\Fr_q))\big)\nonumber\\
    &\overset{\ref{eq:traza_mod_3}}{\equiv}
      \tr\big(\rhot(\Fr_q)\big) \Mod{1+\sqrt{-2}}\nonumber\\
    \therefore\;\; a_q(g)
    &\equiv\tr\big(\rhot(\Fr_q)\big) \Mod{\fP},\label{eq:cong_coef_fourier}
  \end{align}
  porque $(1+\sqrt{-2})\subseteq\fP$.


  A primera vista parece que tenemos las condiciones suficientes de la proposici\'on \ref{prop:cond_mod_rhop} para concluir que $\rhot$ es modular. Pero bajo mejor inspecci\'on observamos que el peso de la forma primitiva $g$ es 1, en lugar de 2. Entonces el siguiente paso es subir el peso de $g$ a 2 multiplic\'andola por una serie de Eisenstein.

En particular tomamos la serie de Eisenstein $E_{1,\psi}$ de peso 1 definida por
  \[
    E_{1,\psi}(z)=1+6\sum_{n=1}^{\infty}\sum_{d\mid n}\psi(d)e^{2\pi inz},
  \]
donde $\psi$ es el caracter de Dirichlet impar m\'odulo 3, i.e. el s\'imbolo de Legendre:
  \[
    \psi(d)=\paren{\frac{d}{3}}=
    \begin{cases}
      1 &d\equiv1\Mod{3}\\
      -1&d\equiv -1\Mod{3}\\
      0&d\equiv0\Mod{3}
    \end{cases}.
  \]
  La raz\'on por la cual tomamos a esta serie de Eisenstein en particular es que cumple las siguientes dos propiedades, la segunda siendo trivial:
  \begin{equation}
    \label{eq:prop_eisenstein_peso1}
    E_{1,\psi}\in M_1(\Gamma_0(3),\psi)
    \quad\mathrm{y}\quad a_n(E_{1,\psi})\equiv
    \begin{cases}
        1 \Mod{3} & n=0\\
        0 \Mod{3} & n>0
      \end{cases}.
    \end{equation}
    El hecho que $E_{1,\psi}$ es modular no es trivial (v\'ease el ejercicio 9.6.4 de \cite{DiamondShurmanAFCIMF}). Otra manera de probar la modularidad de $E_{1,\psi}$ es viendo que $E_{1,\psi}$ es la transformada de Mellin inversa de $\zeta(s)\zeta(s,\psi)$ \cite{Wiles}.%\marginpar{checar esto y cita}
    
    Recuerde que $M(\Gamma_0(N))=\bigoplus M_k(\Gamma_0(N))$ es un anillo graduado por el peso y contiene al ideal $S(\Gamma_0(N))=\bigoplus S_k(\Gamma_0(N))$ (c.f. la proposici\'on \ref{prop:prop_de_M}.\ref{prop_de_M_3}. Como $E_{1,\psi}\in M_1(\Gamma_0(3),\psi)\subset M_1(\Gamma_0(3N))$ (v\'ease los primeros tres p\'arrafos de la secci\'on \ref{sec:formas_primitivas}) y como $g\in S_1(\Gamma_0(N),\chi)\subset S_1(\Gamma_0(3N))$, entonces $gE_{1,\psi}\in S_2(\Gamma_0(3N))$; denotamos $f:=gE_{1,\chi}$.
    
    Como el nebentypus de $g$ es $\chi$ y el nebentypus de $E_{1,\psi}$ es $\psi$, tenemos que $f\in S_2(\Gamma_0(3N),\chi\psi)$. En particular $\gen{d}f=\chi(d)\psi(d)f$ (cf. la proposici\'on \ref{prop:chieigenespacios}) o a nivel de coeficientes de Fourier:
    \begin{equation}\label{eq:diamondf}
      a_n(\gen{d}f)=\chi(d)\psi(d)a_n(f)\qquad\forall d\in(\ZZ/3N\ZZ)^*,\; n>0.
    \end{equation}
Además, $g$ y $E_{1,\psi}$ están normalizadas, entonces $f$ está normalizada, i.e. $a_1(f)=1$. Los dem\'as coeficientes de Fourier de $f$ se pueden calcular módulo 3 con \eqref{eq:prop_eisenstein_peso1}:
  \[
    a_n(f)=
    a_n(g)+\sum_{\underset{i,j>0}{i+j=n}}a_i(g)a_j(E_{1,\chi})\equiv
    a_n(g) \Mod{3}\qquad(\forall n>1),
  \]
que también es válida para $n=1$. Es decir
  \begin{equation}\label{eq:cong_coef_fourier_gE}
  	a_n(f)\equiv a_n(g) \Mod{3}\qquad(\forall n>0).
  \end{equation}
Si juntamos esta congruencia con \eqref{eq:cong_coef_fourier}, obtenemos que para casi todo primo $q$ se tiene
  \begin{equation}\label{eq:cong_coef_prod}
    a_q(f)\equiv\tr\big(\rhot(\Fr_q)\big) \Mod{\fP}.
  \end{equation}
  
  Otra vez parece que estamos en posici\'on de aplicar la proposici\'on \ref{prop:cond_mod_rhop} para concluir que $\rhot$ es modular pero inmediatamente vemos que $f$ no necesariamente es forma primitiva. Por la elecci\'on de $E_{1,\psi}$ tenemos que, aunque $f$ no sea una forma primitiva genuina, s\'i es una ``eigenforma m\'odulo $\fP$''. M\'as precisamente, mediante la asignaci\'on de la serie de Fourier, podemos pensar a $g$ como elemento del anillo de series de potencias formales $\Oo_g[[e^{2\pi iz}]]$. Similarmente $E_{1,\psi}\in\ZZ[[e^{2\pi iz}]]\subset\Oo_g[[e^{2\pi iz}]]$. Bajo esta interpretaci\'on, la congruencia \eqref{eq:cong_coef_fourier_gE} se reescribe como
  \begin{equation}\label{eq:g_cong_gE}
    f\equiv g \Mod{\fP[[e^{2\pi iz}]]},
  \end{equation}
  ya que $(3)\subset\fP\subset\Oo_g$. Entonces si aplicamos un operador de Hecke $T_n$, donde $(n,3N)=1$, a ambos lados la congruencia se preserva. En efecto, por la proposici\'on \ref{prop:coefTnf} y la f\'ormula \ref{eq:diamondf}, los coeficientes de $T_n(f)$ cumplen:
  \begin{align*}
    a_m(T_nf)
    &=\sum_{d\mid (m,n)}d \chi(d)\psi(d)a_{nm/d^2}(f)\overset{\eqref{eq:cong_coef_fourier_gE}}{\equiv}\sum_{d\mid (m,n)}d \chi(d)\psi(d)a_{nm/d^2}(g)\Mod{3}\\
    &\equiv \sum_{d\mid (m,n)}d^2 \chi(d) a_{nm/d^2}(g)\equiv\sum_{d\mid (m,n)}\chi(d) a_{nm/d^2}(g)\\
    \therefore\;\; a_m(T_nf)
    &\equiv a_m(T_ng)\Mod{3},
  \end{align*}
  donde hemos usado que $\psi(d)\equiv d\Mod{3}$ y $d^2\equiv1\Mod{3}$ ya que $d\mid n$ y $(n,3)=1$. Si usamos la notaci\'on de \eqref{eq:g_cong_gE}, las congruencias anteriores se reescriben como
  \[
    T_nf\equiv T_ng \Mod{\fP[[e^{2\pi iz}]]}\qquad\forall(n,3N)=1.
  \]
  De esta manera: 
  \[
    T_n(f)\equiv T_n(g)=a_n(g)g\equiv a_n(g)f\Mod{\fP[[e^{2\pi iz}]]},
  \]
  donde la igualdad se sigue de que los valores propios de la forma primitiva $g$ son sus coeficientes de Fourier (cf. el teorema \ref{thm:valores_propios_Fourier}). En palabras, los coeficientes de $T_n(f)$ y de $a_n(g)f$ son iguales módulo $\fP$; es a esto a lo que nos referimos cuando decimos que $f$ es una eigenforma ``m\'odulo $\fP$''.
 
  
  El siguiente y \'ultimo paso es aplicar el lema de levantamiento de Deligne-Serre a $f$ para obtener una eigenforma genuina que sea congruente a $f$ m\'odulo $\fP$ para que preserve la congruencia \eqref{eq:cong_coef_fourier} que es necesaria para deducit la modularidad de $\rhot$. Para enunciar el lema, introducimos la notaci\'on: sea $\fO$ un dominio de Dedekind con un ideal maximal $\m$ y cociente $k=\fO/\m$; sean $M$ un $\fO$-m\'odulo libre de rango finito y $\Ff\subseteq\End_{\fO}(M)$ una familia de endomorfismos que conmutan dos a dos. Decimos que dos elementos $h,h'\in M$ son congruentes m\'odulo $\m$, denotado de la manera usual, si sus im\'agenes en $M/\m M$ son iguales.

\begin{lema}\label{lema:deligne_serre}%%%%%%%%%%%%%%%%%%%%%%%%%%%%%%%%%%% LEMA
(Deligne-Serre) Si $f\in M-\{0\}$ es tal que $Tf\equiv a_Tf\Mod{\m}$ para toda $T\in\Ff$, i.e. es un vector propio m\'odulo $\m$ para todo endomorfismo de $\Ff$, entonces existe un dominio de Dedekind $\fO'$ y un ideal primo $\m'\subset\fO'$ tal que $\fO\subseteq\fO'$, $\m=\fO\cap\m'$ y el campo de fracciones de $\fO'$ es una extensi\'on finita del campo de fracciones de $\fO$; adem\'as existe un elemento $f'\in\fO'\otimes_{\fO}M$ distinto de cero tal que $Tf'=a'_{T}f'$ para toda $T\in\Ff$ y tal que $a_T\equiv a'_T\Mod{\m'}$.
\end{lema}%%%%%%%%%%%%%%%%%%%%%%%%%%%%%%%%

\begin{nota}
  El lema original est\'a enunciado para $\fO$ un anillo de valoraci\'on discreta pero la prueba es f\'acilmente adaptada para dominios de Dedekind porque localmente, éstos son anillos de valoración discretas.  
\end{nota}

Aplicamos el lema con $\fO=\Oo_g$, $\m=\fP$, $M=S_2(\Gamma_0(3N),\chi\psi)$, $\Ff=\{T_n\mid (n,3N)=1\}$ y $f=gE_{1,\psi}$. Obtenemos una extensi\'on de  anillos $\Oo_g\subseteq\Oo$, un ideal primo $\fP'\subset\Oo$ tal que $\fP=\fP'\cap\Oo_g$ y un elemento $f'\in \Oo\otimes S_2(\Gamma_0(3N),\chi\psi)$ que es eigenforma para todo operador de Hecke fuera de $3N$ cuyos valores propios $a'_{T_n}\in\Oo$ cumplen:
\begin{equation}\label{eq:valores_propios_fprima}
  a'_{T_n}\equiv a_n(f) \Mod{\fP'}.
\end{equation}
Como $f'$ es eigenforma sus valores propios son sus coeficientes de Fourier (cf. el teorema \ref{thm:valores_propios_Fourier}). Adem\'as, como $\fP\subset\fP'$, podemos juntar las congruencias \eqref{eq:cong_coef_prod} y \eqref{eq:valores_propios_fprima} para concluir que para casi todo primo $q$ (que adem\'as cumple $q\nmid N$) tenemos:
\[
  a_q(f')=a'_{T_q}\equiv a_q(f)\equiv\tr(\rhot(\Fr_q))\Mod{\fP'}.
\]
Finalmente tenemos las condiciones suficientes para aplicar la proposici\'on \ref{prop:cond_mod_rhop} para concluir que $\rhot$ es modular. Por lo tanto lo \'ultimo que falta es probar el lema de levantamiento de Deligne-Serre que hacemos a continuaci\'on.
\end{proof}

\begin{proof}[Demostraci\'on del lema \ref{lema:deligne_serre}]
  Sea $\Hh$ la $\fO$-sub\'algebra de $\End_{\fO}(M)$ generada por $\Ff$. Como $M$ es libre de rango finito, entonces $\End_{\fO}(M)$ es libre de rango finito, y as\'i $\Hh$ es un $\fO$-m\'odulo libre de rango finito, en particular es un m\'odulo plano\footnote{Un $\fO-$m\'odulo $\Hh$ es plano si el funtor $N\mapsto N\otimes\Hh$ es exacto izquierdo (recuerde que este funtor siempre es exacto derecho). Gracias a que el producto tensorial y la suma directa conmutan, todo m\'odulo libre es plano.}.

  Ahora, definimos $\eps:\Hh\ra k$ como el morfismo de $\fO-$álgebras que asigna valores propios, es decir definimos $\eps$ sobre los generadores de $\Hh$ como
  \[
    \eps(T):=a_T+\m\qquad(\forall T\in\Ff)
  \]
Observa que  por construcci\'on $\eps|_{\fO}=\Id_{\fO}$, entonces $\eps$ es sobreyectivo. Por lo tanto $\Hh/\ker\eps\cong k$ y as\'i $\ker\eps\subset\Hh$ es un ideal maximal.

Sea $\p\subseteq\ker\eps$ un ideal primo minimal. La existencia de primos minimales del anillo se sigue de la existencia de conjuntos multiplicativamente cerrados maximales. M\'as precisamente, si $A$ es cualquier anillo y $\Sigma$ es la familia de subconjuntos de $A$ multiplicativamente cerrados que no contienen al 0, entonces por el lema de Zorn, $\Sigma$ tiene elementos maximales y adem\'as $S\in\Sigma$ es maximal si y solo si $A-S$ es un ideal primo minimal con respecto de otros ideales primos (v\'ease el ejercicio 3.6 de \cite[\S3]{AtiyahCA}). Por lo tanto si aplicamos estos resultados a la localizaci\'on de $\Hh$ en el ideal $\ker\eps$, concluimos que existen ideales primos minimales contenidos en $\ker\eps$.

Como $\p$ es minimal, todo sus elementos distintos de cero son divisores de cero. En efecto: si denotamos al conjunto de divisores del cero junto con el mismo 0 por $D$ y suponemos que $\p\not\subseteq D$ entonces $\Hh-D\not\subseteq\Hh-\p$; tomamos $h\in\Hh-D$ tal que $h\not\in\Hh-\p$. Como $1\in\Hh-\p$ concluimos que $h=h\cdot1\in(\Hh-D)(\Hh-\p)$ y as\'i el conjunto multiplicativamente cerrado $(\Hh-D)(\Hh-\p)$ contiene estictamente al conjunto multiplicativo maximal $\Hh-\p$. Esto es una contradicci\'on. Por lo tanto $\p\subseteq D$.

  Como $\Hh$ es un $\fO$-m\'odulo libre, para toda $x\in\fO$ el endomorfismo $h\mapsto xh$ de $\Hh$ se representa por la matriz diagonal $x\Id_M$ cuyo determinante es una potencia de $x$ que (salvo en el caso $x=0$) es distinto de cero porque $\fO$ es un dominio entero. En particular $h\mapsto xh$ es inyectiva para toda $x\in\fO-\{0\}$. Por lo tanto $\fO$ no tiene divisores de cero en $\Hh$ y as\'i $\p\cap\fO=0$.

  De esta manera la composici\'on $\fO\ra\Hh\epi\Hh/\p$ es inyectiva; por lo tanto podemos considerar a $\fO$ como un subanillo de $\Hh/\p$. Adem\'as, como $\Hh$ es un $\fO$-m\'odulo finitamente generado, entonces $\Hh/\p$ tambi\'en es un $\fO$-m\'odulo finitamente generado. Como $\fO$ es un anillo noetheriano (por ser dominio de Dedekind), entonces $\Hh/\p$ es un $\fO$-m\'odulo noetheriano, i.e. todos sus subm\'odulos son finitamente generados (v\'ease por ejemplo la proposici\'on 1.4 de \cite{EisenbudCA})

  Este comentario sirve para probar que $\Hh/\p$ es una extensi\'on entera de $\fO$. En efecto, si tomamos $T+\p\in\Hh/\p$ arbitrario, entonces como $\fO[T+\p]=\fO[T]+\p\subseteq\Hh/\p$, tenemos que $\fO[T+\p]$ es un $\fO$-m\'odulo finitamente generado para toda $T+\p\in\Hh/\p$. Esto es una condici\'on equivalente a ser entero sobre $\fO$  (cf. la proposici\'on 5.1 de \cite{AtiyahCA}), por lo tanto $T+\p$ es entero sobre $\fO$ para toda $T+\p\in\Hh/\p$.

  Ahora, sea $L$ el campo de fracciones del dominio entero $\Hh/\p$ y $\fO_L$ la cerradura entera de $\fO$ en $L$. Esto hace que $\fO_L$ sea un dominio de Dedekind (cf. la proposici\'on 8.1 de $\S1.8$ en \cite{NeukirchANT}). Como $\fO\subseteq\Hh/\p$ es una extensi\'on entera, tenemos que $\Hh/\p\subseteq\fO_L$. Con esto definimos $\delta:\Hh\ra\fO_L$ como la composici\'on de $\Hh\epi\Hh/\p\hookrightarrow\fO_L$ y denotamos $a'_T:=\delta(T)$ para toda $T\in\Ff$. Resumimos estos dos p\'arrafos con el siguiente diagrama conmutativo:
  \[
    \begin{tikzcd}
      &\Hh \arrow[d,two heads] \arrow[dr,"\delta"]&\\
      \fO \arrow[ur] \arrow[r,hook] & \Hh/\p \arrow[r,hook] & \fO_L.
    \end{tikzcd}
  \]

  Como $\ker\eps\subset\Hh$ es un ideal maximal que contiene a $\p$, entonces $\ker\eps+\p\subset\Hh/\p$ es un ideal maximal. Sea $\m'\subset\fO_L$ un ideal primo divisor del ideal $(\ker\eps+\p)\fO_L$, en particular $\m'\cap\Hh/\p=\ker\eps+\p$ y adem\'as, como $\fO_L$ es un dominio de Dedekind, $\m'$ tambi\'en es maximal. Por lo tanto, del diagrama anterior tenemos
  
  \begin{equation}\label{eq:contraccion_m}
	\delta(\ker\eps)\subseteq\m'.  
  \end{equation}

  Esto \'ultimo nos garantiza que $a'_T\equiv a_T\Mod{\m'}$, porque las igualdades $\eps(T-a_T\Id_M)=\eps(T)-a_T+\m=0+\m$ para toda $T\in\Ff$ implican que $T-a_T\Id_M\in\ker\eps$ y por lo anterior tenemos que:
  \begin{equation}\label{eqn:eigenvalorescoincidenmodm}
    \delta(T-a_T\Id_M)=a'_T-a_T\in\m' \quad\then\quad a'_T\equiv a_T\Mod{\m'}
  \end{equation}
 
  Con esto sabemos quienes tienen que ser los valores propios, ahora tenemos que construir un vector propio con esos valores propios. Como $\Hh$ es un $\fO$-m\'odulo plano, entonces la inclusi\'on $\fO\hookrightarrow L$ se preserva cuando tomamos el producto tensorial con $\Hh$, es decir tenemos una inclusi\'on
  \[
    \Hh\cong\fO\otimes_{\fO}\Hh\hookrightarrow L\otimes_{\fO}\Hh.
  \]
Observe que $L\otimes M$ es un $L\otimes\Hh$-m\'odulo finitamente generado con la acci\'on $(\la\otimes T)(\mu\otimes f)=(\la\mu\otimes Tf)$. En efecto, $M$ es finitamente generado y libre sobre sobre $\Oo$, entonces es finitamente generado sobre $\Hh$. Como hacer producto tensorial con $L$ conmuta con la suma directa, $L\otimes M$ es finitamente generado sobre $L\otimes\Hh$.

Sea $\fP\subseteq L\otimes\Hh$ el ideal generado por la imagen de $\p$ bajo la inclusi\'on $\Hh\subset L\otimes\Hh$ (note que $\fP$ no necesariamente es primo). Como $\Hh$ es un $\fO$-m\'odulo noetheriano, $\p$ es un ideal finitamente generado por algunas $\{T_1,\dots,T_n\}\subset\p$. Por lo tanto $\fP$ es un ideal de $L\otimes\Hh$ finitamente generado por $\{1\otimes T_1,\ldots,1\otimes T_n\}$. Como $\p$ consta de puros divisores de cero, existen $T'_1,\ldots,T'_n\in\Hh-\{0\}$ tales que $T_iT'_i=0$ para toda $i=1,\ldots,n$. Adem\'as, para cada $1\otimes T_i\in\fP$ toma un $f_i\in M$ tal que $T'_i(f_i)\neq 0$. De esta manera
\[
  (1\otimes T_i)(1\otimes T'_i(f_i))=(1\otimes T_i(T'_i(f_i)))=1\otimes 0=0.
\]
Por lo tanto todos los generadores de $\fP$ son divisores de cero de $L\otimes M$ como $L\otimes\Hh$-m\'odulo y as\'i $\fP\subseteq D'$ donde $D'$ es el conjunto de divisores de cero de $L\otimes M$.

  El conjunto de los divisores de cero de un m\'odulo finitamente generado (junto con el cero) es la uni\'on de los ideales primos asociados\footnote{Un ideal primo $\p$ de un anillo $A$ es asociado a un $A$-m\'odulo $M$ si existe un elemento $f\in M$ tal que $\p=(f:0):=\{a\in A \mid af=0\}$.} al m\'odulo \cite[teorema 3.1, pg 89]{EisenbudCA}. Por lo tanto si denotamos al conjunto de ideales primos asociados de $L\otimes M$ como $\Aa=\mathrm{Ass}_{L\otimes\Hh}(L\otimes M)$ tenemos que
\[
  \fP \subseteq D'=\bigcup_{\q\in \Aa}\q.
\]
Por el teorema de ``Prime Avoidence'' (v\'ease por ejemplo la proposici\'on 1.11 de \cite{AtiyahCA}), $\fP$ est\'a contenido en alg\'un $\q\in\Aa$. Por lo tanto existe un elemento $h=\sum_i\mu_i\otimes h_i\in L\otimes M-\{0\}$ tal que $\q$ es su anulador, es decir $\fP\subseteq\q=(h:0)$.

Sea $T\in\Hh$, tengo que probar $\Hh\cong\p\oplus\Hh/\p$, asi $T=x+\delta(T)\Id$ (usa el diagrama conmutativo para calcular el factor de $\Hh/\p$). Cuando extendemos a $T:L\otimes M\ra L\otimes M$ obtenemos $T=X+\delta(T)\Id_{L\otimes M}$. Aplicamos $T$ a $h$ y ya.
\end{proof}