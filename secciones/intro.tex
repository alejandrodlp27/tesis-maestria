\section{Introducci\'on}

\subsection*{Panorama hist\'orica}

El prop\'osito de esta tesis es describir la prueba de un caso particular, pero muy importante,
de la conjetura de Shimura-Taniyama-Weil (STW). La fama de STW claramente viene de su rol en la
prueba del \'Ultimo Teorema de Fermat (UTF) que dice: para $n>2$ tenemos
\begin{equation}\tag*{$\Big[\mathrm{UTF}(n)\Big]$}
  \exists x,y,z\in\ZZ\;\;\text{tales que}\;\; x^n+y^n=z^n\quad\then\quad xyz=0.
\end{equation}
Claramente si $d\mid n$, entonces UTF($d$) $\then$ UTF($n$). Esto quiere decir que solamente hay
que considerar los casos cuando $n=p$ un primo impar; el caso $n=4$ fue probado por el mismo Pierre
de Fermat (1607-1665) cuando demostr\'o que la ecuaci\'on $x^4-y^4=z^2$ no tiene soluciones enteras.

Hasta mediados del siglo XIX, algunos casos particulares de UTF se fueron probando:
Euler prob\'o el UTF para $n=3$ en 1753, Dirichlet y Legendre ambos probaron el caso $n=5$ en los
1820's y en 1839 Lam\'e prueba el caso $n=7$. Ocho a\~nos despu\'es, Lam\'e present\'o una
prueba completa del UTF, pero result\'o estar equivocada pues hab\'ia asumido, incorrectamente,
que el anillo de enteros $\ZZ[e^{2\pi i/p}]$ era un dominio de factorizaci\'on \'unica para todo
primo $p$, pero esto no es cierto (e.g. $p=23$). Usando estas ideas, Kummer prob\'o el UTF para
todo primo regular.\footnote{Un primo $p$ es \emph{regular} si $p\nmid h_K$ donde $h_K$ es el
  n\'umero de clase del campo $K=\QQ(e^{2\pi i/p})$, i.e. el orden del grupo de Picard de
  Spec$(\Oo_K)$.}

Todo cambi\'o cuando Frey sugiri\'o una nueva alternativa en los 80's. Para ese entonces la
geometr\'ia algebr\'aica estaba bien fundamentada y ofrec\'ia herramientas poderosas para estudiar
el UTF. En el \'area particular de curvas el\'ipticas, ya se hab\'ia formado una conjetura
importante:
\begin{equation}\tag*{$\big[\mathrm{STW}\big]$}
  \text{Toda curva el\'iptica sobre}\;\QQ\;\text{es modular.}
\end{equation}
Frey sugiri\'o que de un contraejemplo $a^p+b^p=c^p$ de UTF($p$), la curva el\'iptica asociada
\[
  E_{a,b,c,p}:\; y^2=x(x-a^p)(x+b^p).
\]
podr\'ia ser un contraejemplo de STW. Esta curva se llama la \emph{curva de Frey} en su honor
apesar de que la conexi\'on entre la curva de Frey y el UTF fue establecido por Hellegouarch
unos a\~nos antes.

Para argumentar porqu\'e $E=E_{a,b,c,p}$ podr\'ia contradecir STW, Frey, junto con Serre, describieron
las propiedades de las representaciones $\bar{\rho}=\rhop$ de Galois asociadas a los puntos de
$p-$torsi\'on de $E$. En particular, ellos probaron que $\bar{\rho}$ era impar, absolutamente
irreducible, no-ramificado fuera de $2p$ y plano sobre $p$. Cumplir al mismo tiempo estas
cuatro propiedades es excepcional para una representaci\'on de Galois y sugiere fuertemente que
tal $\bar{\rho}$ no puede existir.

Serre formul\'o expl\'icitamente varias conjeturas sobre c\'omo clasificar representaciones
de Galois, e.g. $\bar{\rho}$, seg\'un la teor\'ia de formas modulares. M\'as precisamente,
estudi\'o c\'omo asociar representaciones $\rho_f$ a ciertas formas modulares $f$ y cuando
pasaba que una representaci\'on arbitraria $\rho$ era de la forma $\rho=\rho_f$, i.e. cuando
$\rho$ era modular. En particular, Serre conjetur\'o que a las representaciones modulares $\rho_f$
que adem\'as cumpl\'ian las propiedades extraordinarias de $\bar{\rho}$, se les pod\'ia bajar su
\emph{nivel} hasta su conducto de Artin.

La reducci\'on de nivel de (ciertas) representaciones modulares lo probaron Ribet y Mazur en los
80's y por fin la intuici\'on de Frey se confirm\'o: el conductor de Artin de $\bar{\rho}$ es 2,
entonces si STW fuese cierto y $E$ fuese modular, la representaci\'on $\bar{rho}$ ser\'ia modular
y por el teorema de Mazur-Ribet induce una representaci\'on modular de nivel 2 asociado a una
forma modular cuspidal $f$ de nivel 2 no trivial, pero era bien conocido que el espacio de
tales formas modulares es nulo; contradicci\'on. Por lo tanto la curva de Frey era un
contraejemplo de STW. El camino a la prueba del UTF se ilumin\'o: pruebas la conjetura de
Shimura-Taniyama-Weil y pruebas el \'Ultimo teorema de Fermat.

Esquem\'aticamente, la prueba del UTF se ve as\'i:\\


\begingroup
\centering
\resizebox{\textwidth}{!}{
\begin{tikzpicture}[
  nonterminal/.style={rectangle,rounded corners=3mm,minimum size=6mm,line width=2pt,draw=black!80,
    top color=white,bottom color=black!20},
  terminal/.style={rectangle,rounded corners=3mm,minimum size=6mm,very thick,draw=black!50,
    top color=white,bottom color=black!20},
  noborder/.style={rectangle,rounded corners=3mm,minimum size=6mm,draw=white},
  point/.style={circle,inner sep=-1pt,minimum size=1pt,fill=black},
  thm/.style={rectangle,draw=red!50,top color=white,bottom color=red!20,line width=2pt},
  lin/.style={line width=2pt,draw=black!70,font=\scriptsize},
  node distance=15mm,align=center]

  \matrix[column sep=20mm,row sep=0mm,ampersand replacement=\&]{
  \&\&\node(wiles) [nonterminal] {$E$ es modular, i.e.\\ $\exists f\in S_2(\Gamma_0(N))$\\ tal que $\bar{\rho}_f\cong\rhop$}; \&[-12mm]\&\&[-12mm] \\
  
  \node (flt) [nonterminal] {$\exists (a,b,c)$ solución\\ no trivial de\\ $x^p+y^p+z^p=0$.}; \&
  \node (frey) [nonterminal] {$\exists E/\QQ$ asociado a $(a,b,c)$\\ que es semiestable y tiene\\ conductor $N$ par}; \&\&[-12mm]
  \node (p) [point]{}; \&
  \node (ribet) [nonterminal] {$\exists g\in S_2(\Gamma_0(2))$\\ tal que $\bar{\rho}_f\cong\bar{\rho}_g$};\&[-12mm]
  \node (contra) [terminal] {$\ra\leftarrow$}; \\
  
  \&\&[-12mm] \node(serre) [nonterminal] {$\rhop$ es no ramificado\\ en $q\nmid 2p$ y plano en $p$}; \&[-12mm]\&\&[-12mm]
  \node (con) [noborder] {dim$S_2(\Gamma_0(2))=0$.};\\
};
\graph{
  (flt) ->[lin,edge label=Frey]
  (frey) ->[lin,edge label=Wiles]
  (wiles) --[lin]
  (p) ->[lin,edge label=Ribet]
  (ribet) ->[lin]
  (contra);
  
  (frey) ->[lin,edge label=Serre]
  (serre) --[lin]
  (p);
  
  (con) -- (contra)
  
};
\end{tikzpicture}
}
\endgroup

\subsection*{Sobre la tesis}





\begingroup
\centering

\resizebox{!}{0.8\textheight}{
\begin{tikzpicture}[
  nonterminal/.style={rectangle,rounded corners=3mm,minimum size=6mm,line width=2pt,draw=black!80,
    top color=white,bottom color=black!20},
  terminal/.style={rectangle,rounded corners=3mm,minimum size=6mm,very thick,draw=black!50,
    top color=white,bottom color=black!20},
  point/.style={circle,inner sep=-1pt,minimum size=1pt,fill=black},
  thm/.style={rectangle,draw=red!50,top color=white,bottom color=red!20,line width=2pt},
  lin/.style={line width=2pt,draw=black!70},
  node distance=15mm,align=center]

\matrix[row sep=10mm, column sep=-10mm,ampersand replacement=\&]{
  \&\node (semis) [nonterminal] {$E$ es semiestable.};\&\\%[-5mm]
  
  \&\node (p) [point]{};\&\\%[-10mm]
  
  \&\&\node (red) [nonterminal] {$\rhot$ es reducible.};\\[5mm]
  
  \&\&\node (cincoirred) [nonterminal] {$\rhoc$ es irreducible.};\\[5mm]
  
  \node (irred) [nonterminal] {$\rhot$ es irreducible.};
  \&\&\node (eprima) [nonterminal] {$\exists E'$ tal que
    $\rhoc\cong\ol{\rho}_{E',5}$,\\ $\ol{\rho}_{E',3}$ es irreducible.};\\

  \&\node (LT) [thm]{Teorema de\\ Langlands-Tunnell};\&\\
  
  \&\&\node (eprimamodular) [nonterminal] {$\ol{\rho}_{E',3}$ es modular.};\\[-5mm]
  
  \node (rhotmodular) [nonterminal] {$\rhot$ es modular.};\&\&\\[-5mm]

  \&\&\node (rhocmodular) [nonterminal] {$\rhoc(\cong\ol{\rho}_{E',5})$ es modular.};\\

  \&\node (emodular) [nonterminal] {$E$ es modular.};\&\\%[-155mm]
};
\graph {
  (semis) --[lin]
  (p) ->[bend right,lin]
  (irred) --[bend left,lin]
  (LT) ->[bend left,lin,shorten >=5pt]
  (rhotmodular) ->[lin,edge label'=CLMS(3), shorten >=2pt] (emodular);
  
  (p) ->[lin,bend left=10]
  (red) ->[lin,edge label=$\S$\ref{sec:X015}]
  (cincoirred) ->[lin,edge label=$\S$\ref{sec:3_5}]
  (eprima) --[lin,bend right]
  (LT) ->[lin,bend right]
  (eprimamodular) ->[lin,edge label=T.\ref{thm:equivmodular}]
  (rhocmodular) ->[lin,bend left,edge label=CLMS(5)]
  (emodular);
};
\end{tikzpicture}
}

\endgroup

