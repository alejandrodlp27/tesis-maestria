\section{Los teoremas de modularidad}

En esta sección describimos los diferentes teoremas sobre la modularidad de una curva elíptica $E/\QQ$. Es común que se junten estos teoremas en uno solo llamado el \emph{teorema de modularidad} que simplemente dice
\begin{thm}(de Modularidad)
	Toda curva elíptica $E/\QQ$ es modular.
\end{thm}
Los diferentes teoremas surgen de las diferentes maneras en que uno puede definir \emph{modular} y se conjeturaron en diferentes momentos. Antes de la prueba completa en 1999 \cite{s}, todas las diferentes conjeturas se conocían por \emph{la conjetura de Shimura-Taniyama-Weil}. El propósito de esta sección es enunciar las diferentes versiones del teorema de modularidad y ver cómo se relacionan entre ellos. El excelente libro \cite{DiamonShurman} está dedicado a describir el teorema de modularidad y ahí referimos al lector para consultar cualquier detalle o prueba omitida. 

Wiles probó el teorema de modularidad para la clase de curvas elípticas semiestables, que gracias al trabajo de Frey Serre y Ribet, es suficiente para probar el UTF. La propiedad de semiestabilidad le permitió probar una versión del teorema de modularidad enunciada con las representaciones de Galois asociadas a la curva. Esta riqueza en diferentes áreas de la matemática es lo que permitió probar el UTF. 
